%% LyX 2.3.6.2 created this file.  For more info, see http://www.lyx.org/.
%% Do not edit unless you really know what you are doing.
\documentclass[oneside,english]{amsart}
\usepackage[T1]{fontenc}
\usepackage{geometry}
\geometry{verbose,tmargin=1in,bmargin=1in,lmargin=1in,rmargin=1in}
\usepackage{amstext}
\usepackage{amsthm}
\usepackage{graphicx}

\makeatletter
%%%%%%%%%%%%%%%%%%%%%%%%%%%%%% Textclass specific LaTeX commands.
\numberwithin{equation}{section}
\numberwithin{figure}{section}
\theoremstyle{plain}
\newtheorem{thm}{\protect\theoremname}
\theoremstyle{definition}
\newtheorem{example}[thm]{\protect\examplename}
\theoremstyle{definition}
\newtheorem{xca}[thm]{\protect\exercisename}

\makeatother

\usepackage{babel}
\providecommand{\examplename}{Example}
\providecommand{\exercisename}{Exercise}
\providecommand{\theoremname}{Theorem}

\begin{document}
\title{WCSAM 206 11 - Intro to Binary and ASCII (1937-1960s CE)}
\maketitle

\section{Last Time}
\begin{itemize}
\item Enigma consists of three components in the following order:
\[
P\mapsto S\to W\to UKW\to W\to S\mapsto C
\]
where $S$ is the plugboard (steckerbrett), $W$ is the rotors (walzen),
and $UKW$ is the reflector (umkehrwalze).
\item Plugging two letters together in the plugboard swaps those letters
with each other. So if a and w are plugged, we have the plugboard
setting $(aw)$, meaning that every a that goes through the plugboard
comes out as a w and vice versa.
\item There are 3-4 rotors that, together, encode a different cipher alphabet
for each letter. This alphabet changes for each letter since entering
one letter shifts the rotors.
\item The reflector sends the electric signal that enters it back through
the rotors and then through the plugboard. The function of the reflector
is to make it so that typing in plaintext into Enigma outputs ciphertext,
while typing in ciphertext outputs plaintext.
\end{itemize}
\begin{example}
{[}Canvas discusison{]} Illustrate with the sample message how to
encrypt/decrypt.
\end{example}

\begin{xca}
{[}Canvas discussion{]} Encrypt a message using some Enigma settings,
then post your message for others to decrypt.
\end{xca}


\section{Reading question: early computer cryptography}
\begin{xca}
Read \emph{The Code Book }from the beginning of Ch. 6 to the heading
``God Rewards Fools'', then answer the following questions:
\begin{enumerate}
\item What was the occasion for the building of the world's first programmable
computer?
\begin{enumerate}
\item The British invented the first programmable computer, named Colossus,
in order to break the Lorenz cipher used to encrypt communications
between Hitler and his generals.
\end{enumerate}
\item What are three significant differences between computer encryption
and the mechanical encryption used in Enigma?
\begin{enumerate}
\item A computer can mimic the action of cipher machines that are too complex
to actually be built.
\item A computer can operate much more quickly than mechanical scramblers.
\item Computers scramble numbers rather than letters of the alphabet.
\end{enumerate}
\item Mimic the way a computer would encrypt the message ``yes'' by doing
the following, then bring your work and ciphertext to class on separate
sheets of paper on the date below:
\begin{enumerate}
\item {[}slide{]} Using Table 24 in \emph{The Code Book}, convert ``yes''
to ASCII.
\begin{enumerate}
\item 121 101 115
\end{enumerate}
\item Choose a message key that's three letters long or less, and convert
that key into ASCII.
\begin{enumerate}
\item e.g. ``woo'' = 119 111 111
\end{enumerate}
\item Add the ASCII digits of ``yes'' to the ASCII digits of your key
to perform an ASCII Vigen\`{e}re cipher!
\end{enumerate}
\end{enumerate}
\end{xca}

\begin{xca}
Pass your ciphertext clockwise in your groups, telling the recipient
your keyword (but not your key numbers). Decrypt the message you've
received, converting the ASCII back to text.
\end{xca}


\section{Binary Codes}
\begin{xca}
(Reading Question: Early Computer Cryptography) Please read Sections
3.2 through 3.4 of your Coursepack, then give the following questions
from p61-62 (bottom numbering) of your Coursepack your best attempt,
showing all work. Then turn your work in on paper on the date below.
\begin{itemize}
\item \#4(a,b): Convert the following decimal integers to binary.
\begin{enumerate}
\item $13$: since $13=8+4+1=1(2^{3})+1(2^{2})+0(2^{1})+1(2^{0})$, we have
$13=1101_{2}$.
\item $45$: since 
\[
45=32+13=32+8+4+1
\]
we can write
\begin{align*}
45 & =2^{5}+2^{3}+2^{2}+2^{0}\\
 & =101101_{2}.
\end{align*}
\end{enumerate}
\item \#5(a,b): convert the binary integers to decimal.
\begin{enumerate}
\item $10000_{2}$: 
\[
10000_{2}=1\times2^{4}+0=16
\]
\item $101001_{2}$
\begin{align*}
101001_{2} & =1\times2^{5}+1\times2^{3}+1\times2^{1}\\
 & =32+8+1=41.
\end{align*}
\end{enumerate}
\item \#8
\begin{itemize}
\item We add $\mod2$ to get $10011000110100_{2}$.
\end{itemize}
\end{itemize}
\end{xca}

\textbf{Today. }We'll discuss the way computers encode information
and the unique challenges computers pose for cryptography. Then we'll
talk about what some call the most important cryptological development
since the Caesar cipher!
\begin{itemize}
\item Computers store information as \textbf{bits}, $0$s and $1$s. This
is called ``binary'' or ``base 2''. They do this because information
is stored in a binary format (e.g. high and low voltages) that we
represent with binary digits (or bits) $0$ (off) and $1$ (on).
\item So a number like $4085$, as stored in your computer, looks like $1111111110101_{2}$.
The subscript $2$ refers to the \textbf{base}, or the number whose
powers form the ``places'' in your number.
\item \textbf{How?} Well, normally, we write numbers ``base 10''. So a
number like $4085$ really means
\begin{align*}
4085 & =4(10^{3})+0(10^{2})+8(10^{1})+5(10^{0})\\
 & =4000+0+80+5.
\end{align*}
\item Throughout history, people have used other bases. The ancient Babylonians
used $60$ as a base, so they'd write $4085=185_{60}$:
\begin{align*}
4085 & =1(60^{2})+8(60)+5\\
 & =3600+480+5.
\end{align*}
\end{itemize}
\begin{xca}
Write the following binary integers in base $10$ (our standard number
system):
\begin{enumerate}
\item $10_{2}$

$=1(2^{1})+0(2^{0})=2$
\item $100_{2}$

$=1(2^{2})+0(2^{1})+0(2^{0})=4$
\item $1111_{2}$

$=1(2^{2})+4(2^{1})+3(2^{0})=4+8+3=15$
\item $110100_{2}$

$=1(2^{5})+1(2^{4})+1(2^{2})=32+16+4=52$
\end{enumerate}
\end{xca}

\includegraphics[scale=0.75]{\string"Old Cryptography Notes/pasted141\string".png}

\section{ASCII}
\begin{itemize}
\item Every single file on a computer is stored as bits (binary digits).
\item To store English text as bits, we just need to associate letters (or
characters) with positive integers and then let the computer store
the integers in a binary format.
\item {[}slide{]} A popular way that computers represent characters as integers
is via the American Standard Code for Information Interchange (ASCII)
as shown:
\end{itemize}
\begin{xca}
Encode ``HI MOM'' in ASCII, then encode the ASCII numbers in binary.

We first convert ``HI MOM'' to ASCII: 
\[
72\text{ }73\text{ }32\text{ }77\text{ }79\text{ }77
\]

We then convert each ASCII decimal to binary:
\begin{align*}
72 & =64+8\\
 & =2^{6}+2^{3}\\
 & =1\times2^{6}+0\times2^{5}+0\times2^{4}+1\times2^{3}+0\times2^{2}+0\times2^{1}+0\times2^{0}\\
 & =1001000_{2}.
\end{align*}
Similarly, since $73=72+1\times2^{0}$, 
\[
73=1001001_{2}.
\]
Since $32=2^{5}$, $32=100000_{2}$. Since $77=73+2^{2}$, 
\[
77=1001101_{2}.
\]
Since $79=77+2^{1}$, 
\[
79=1001111_{2}.
\]
So our message in binary is
\[
1001000,1001001,100000,1001101,1001111,1001101.
\]

\includegraphics[scale=0.75]{\string"Old Cryptography Notes/pasted149\string".png}
\end{xca}

\begin{itemize}
\item We just need to find the largest power of $2$ less than each number,
then add $1$ to that power. 
\begin{enumerate}
\item For example, 
\begin{align*}
100 & =64+32+4\\
 & =2^{6}+2^{5}+2^{2}
\end{align*}
so our binary representation starts with the $2^{6}$ths place and
goes to the $2^{0}$ths place, for a total of $7$ binary digits.
\item The largest power of $2$ less than $200$ is $128=2^{7}$, so $8$
digits.
\item The largest power of $2$ less than $300$ is $256=2^{8}$, so $9$
digits.
\item The largest power of $2$ less than $500$ is also $256$, so $9$
digits.
\item The largest power of $2$ less than $800$ is $512=2^{9}$, so $10$
digits.
\end{enumerate}
\end{itemize}
\includegraphics[scale=0.75]{\string"Old Cryptography Notes/pasted153\string".png}
\begin{xca}
\includegraphics[scale=0.75]{\string"Old Cryptography Notes/pasted152\string".png}

Which of the ciphers that we've used so far are effective in binary?
\end{xca}

\includegraphics[scale=0.75]{\string"Old Cryptography Notes/pasted154\string".png}

\includegraphics[scale=0.75]{\string"Old Cryptography Notes/pasted151\string".png}

How many potential keys are there for a $4$-bit keyword?
\begin{itemize}
\item A $4$-bit keyword can encode $2^{0},2^{1},2^{2},$ and $2^{3}$rds
places. For each bit, there are $2$ possibilities, $0$ and $1$.
Therefore, for $4$ bits, there are 
\[
2\times2\times2\times2=2^{4}=16
\]
possible keys. This is even less secure than a shift cipher!
\end{itemize}
\begin{xca}
~
\begin{enumerate}
\item Choose one group member to be the sender and one the receiever. (You
can have two receivers.) Then have the sender choose a message to
encrypt and convert it to binary. \textbf{Don't share it with the
receiever.}
\item Agree on an equation (values for $e$ and $n$ in the equation $p^{e}\mod n$).
Use the Diffie-Hellman key exchange with this equation to agree on
a numerical key ($\geq10000$). Have someone in your group be Eve
and try to find your key (probably unsuccessfully for such a large
key).
\item Then represent this numerical key in binary and use it to encrypt
the sender's binary message.
\item Receiver(s): decrypt the message using the key, and convert the message
back to ASCII, then to plaintext.
\end{enumerate}
\end{xca}


\end{document}
