%% LyX 2.3.6.2 created this file.  For more info, see http://www.lyx.org/.
%% Do not edit unless you really know what you are doing.
\documentclass[oneside,english]{amsart}
\usepackage[T1]{fontenc}
\usepackage{geometry}
\geometry{verbose,tmargin=1in,bmargin=1in,lmargin=1in,rmargin=1in}
\usepackage{amsthm}
\usepackage{graphicx}

\makeatletter
%%%%%%%%%%%%%%%%%%%%%%%%%%%%%% Textclass specific LaTeX commands.
\numberwithin{equation}{section}
\numberwithin{figure}{section}
\theoremstyle{plain}
\newtheorem*{question*}{\protect\questionname}

\makeatother

\usepackage{babel}
\providecommand{\questionname}{Question}

\begin{document}
\title{WCSAM 206 10 - Probability \& the Index of Coincidence}

\maketitle
\textbf{Today: }so far, for every code I've given you to break, we've
assumed you knew the way the code was encrypted. When might this assumption
be realistic? When might it not?

Reliable: 
\begin{itemize}
\item One code is the ``gold standard''
\item You know your enemy tends to use only one type of given code (e.g.
Enigma)
\end{itemize}
Unreliable:
\begin{itemize}
\item You're dealing with a message from an unknown sender (e.g. you intercepted
something as a security agent)
\item There are multiple codes in common use
\end{itemize}
\begin{question*}
How can you tell how a given code was encrypted \emph{without being
able to break it}?
\end{question*}
\textbf{Answer}: Use mathematics, specifically probability!

\includegraphics[scale=0.5]{\string"Old Cryptography Notes/pasted74\string".png}

\includegraphics[scale=0.5]{\string"Old Cryptography Notes/pasted75\string".png}

Remember our main (non-mechanical) methods of encipherment are the
simple shift cipher, Vigenere cipher, and affine cipher.
\begin{question*}
What would be the IoC for shift ciphertext?

What about for affine ciphertext?
\end{question*}
It turns out both are around 0.066, just like regular English.

One of the advantages of Vigenere is that it obscures letter frequencies;
it thwarts frequency analysis. Should its IoC be $<$ or $>0.066$?

Less, because it should be even less likely that one letter is more
frequent than another.

How do we calculate the IoC for a given body of text? Need to compute
the probability that two randomly selected letters are the same. Let's
define ``random'' and show how we compute probability:

\section*{Counting}

Do all examples as exercises.

\includegraphics[scale=0.75]{\string"Old Cryptography Notes/pasted76\string".png}

\includegraphics[scale=0.75]{\string"Old Cryptography Notes/pasted77\string".png}

\includegraphics[scale=0.75]{\string"Old Cryptography Notes/pasted78\string".png}

\includegraphics[scale=0.75]{\string"Old Cryptography Notes/pasted79\string".png}

\includegraphics[scale=0.75]{\string"Old Cryptography Notes/pasted85\string".png}

\includegraphics[scale=0.75]{\string"Old Cryptography Notes/pasted86\string".png}

\includegraphics[scale=0.75]{\string"Old Cryptography Notes/pasted80\string".png}

\includegraphics{\string"Old Cryptography Notes/pasted133\string".png}

\includegraphics{\string"Old Cryptography Notes/pasted134\string".png}

\includegraphics[scale=0.75]{\string"Old Cryptography Notes/pasted81\string".png}

\includegraphics[scale=0.75]{\string"Old Cryptography Notes/pasted82\string".png}

\includegraphics[scale=0.75]{\string"Old Cryptography Notes/pasted83\string".png}

\includegraphics[scale=0.75]{\string"Old Cryptography Notes/pasted84\string".png}

\includegraphics{\string"Old Cryptography Notes/pasted135\string".png}

\includegraphics{pasted2}

\includegraphics{pasted3}

\includegraphics{pasted4}

\includegraphics{pasted5}

\includegraphics{pasted6}

\includegraphics[scale=0.75]{\string"Old Cryptography Notes/pasted89\string".png}
\end{document}
