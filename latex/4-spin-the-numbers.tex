%% LyX 2.3.6.2 created this file.  For more info, see http://www.lyx.org/.
%% Do not edit unless you really know what you are doing.
\documentclass[oneside,english]{amsart}
\usepackage[T1]{fontenc}
\usepackage{geometry}
\geometry{verbose,tmargin=1in,bmargin=1in,lmargin=1in,rmargin=1in}
\usepackage{babel}
\usepackage{amstext}
\usepackage{amsthm}
\usepackage{graphicx}
\usepackage[unicode=true]
 {hyperref}

\makeatletter
%%%%%%%%%%%%%%%%%%%%%%%%%%%%%% Textclass specific LaTeX commands.
\numberwithin{equation}{section}
\numberwithin{figure}{section}
\theoremstyle{plain}
\newtheorem{thm}{\protect\theoremname}
\theoremstyle{definition}
\newtheorem{xca}[thm]{\protect\exercisename}

\makeatother

\providecommand{\exercisename}{Exercise}
\providecommand{\theoremname}{Theorem}

\begin{document}
\title{WCSBS 3 - Spinning the Numbers}
\maketitle

\subsection{}

\section{Reading questions: is math everywhere?}
\begin{xca}
Please read this Scientific American blog post by math historian Michael
Barany (Links to an external site.)and this response from mathematician
Anna Haensch (Links to an external site.), then answer the questions
below:
\begin{enumerate}
\item What arguments does Barany give against the statement \textquotedbl math
is everywhere\textquotedbl ? How would you reply to his arguments? 
\item What arguments does Haensch give that \textquotedbl math is still
everywhere\textquotedbl ? How would you reply to her arguments? 
\item After reading both articles, what perspective do you have on the way(s)
mathematics interacts with society, power, and oppression? 
\item Haensch argues that Barany's arguments do not support the title of
his blog post. Do you agree? Why or why not? If you agree, how would
you retitle Barany's piece?
\end{enumerate}
\end{xca}


\section{Reading question: what factors affect ACT math achievement?}
\begin{xca}
Each year, the ACT publishes a database, called the Enrollment Management
Database, containing statistics about ACT scores and performance by
state. The 2014-2018 Enrollment Management report can be found here
(Links to an external site.). Play around with the report and examine
it, considering the following questions. Then bring short answers
to each question to class for discussion. Under the \textquotedbl subgroup\textquotedbl{}
dropdown menu on the left sidebar, you can filter by state, income,
race, parents' education, gender, and more. Using the tabs at the
top, you can show these students' outcomes (what kind of college they
preferred, what kind of college they attended, whether they were still
attending college after two years, etc). Note that you can only view
student outcomes like Retention/Transfer if you select a year before
2018, as data is not yet available for the high school class of 2018.
\begin{enumerate}
\item Keeping in mind that this course is concerned with issues of fairness
and justice, what are two or three questions that come to mind when
reading the report? (For example, if the report had included data
on whether students worked in high school, one could ask, \textquotedbl Do
students who work in high school do better or worse on the ACT?\textquotedbl ) 
\item What do you think the statistics in the report tell you about your
questions above? Cite specific statistics and describe why you think
they say this. 
\item What uncertainties do you still have about your question from reading
the report? For example, do you have concerns about whether the statistics
fully answer your question? Are you uncertain how to use the statistics
to answer your questions? 
\item What further information might be necessary in order to answer your
questions?
\end{enumerate}
\end{xca}


\section{Discussion - Growth mindset}
\begin{itemize}
\item Certain experiences cause new connections in the brain to form or
strengthen, making the brain smarter by literally rewiring it. Here\textquoteright s
some evidence:
\begin{itemize}
\item In a study with rats, researchers put some rats in empty cages and
others in stimulating cages with puzzles and other rats. The rats
in the stimulating environments were smarter, and their brains even
weighed more! 
\item London taxi drivers have to give their brains a workout when they
navigate the complicated streets of London. Research suggests this
has an impact on the brain. The part of the brain responsible for
spatial awareness is bigger in taxi drivers compared to other Londoners.
And the longer a person has been a taxi driver, the bigger that part
of the brain. 
\end{itemize}
\item I don't value students based on their academic performance. I respect
trying and failing at least as much as not trying and succeeding.
\end{itemize}
\begin{xca}
\textbf{(P/S Video response: growth mindset and productive failure)}
\begin{enumerate}
\item How would you describe a \textquotedbl growth mindset\textquotedbl{}
in your own words? 
\item What is the value of making mistakes in the learning process? 
\item Name one skill you developed or one thing you learned to do through
hard work and making mistakes. Do you believe that it's possible to
learn math through hard work and making mistakes, or is math somehow
different from music and basketball in this way? Why or why not? 
\item How have your previous experiences in math classes affected your ideas
about how math is learned? Have your previous math classes encouraged
a growth mindset or a fixed mindset? Explain. 
\item What can you do in this class to better encourage mistake-making and
develop a growth mindset? 
\item What can I do in this class to help facilitate your mistake-making
and learning?
\end{enumerate}
\end{xca}


\section{Reading question: Spinning the numbers}
\begin{xca}
Read Ch. 1 (skipping the introduction) of the popular-math book Proofiness:
The Dark Arts of Mathematical Deception (attached as .pdf) from page
1 through the end of the first full paragraph of page 18 (ending \textquotedbl proofiness
to watch out for\textquotedbl ) and briefly answer the following
questions:
\end{xca}

\begin{enumerate}
\item Give one example from real life or the news, other than those given
by Seife, of each of the following types of proofiness: 
\begin{enumerate}
\item A measurement attempting to gauge something that's ill-defined 
\begin{enumerate}
\item IQ
\end{enumerate}
\item A phenomenon which there is no settled-upon way of measuring reliably 
\begin{enumerate}
\item anxiety or depression
\item police discrimination
\item hunger
\item any subjective feeling (Taylor)
\end{enumerate}
\item A number tied to a phony measurement. 
\end{enumerate}
\item Suppose that you're trying to sell a product that has the side effect
of instantly killing one out of every hundred people that tries it.
Using each of Seife's \textquotedbl red flags\textquotedbl{} for
bad measurements, make up three advertising \textquotedbl statistics\textquotedbl{}
to convince people to buy your product and not to worry about its
lethality.
\end{enumerate}

\section{Reading the News discussion}

\textbf{Reading the News: }Part of quantitative literacy is being
able to critically interpret numbers and mathematics in the news media.
Each week, you\textquoteright ll be asked to report on numbers or
mathematics contained in an article you\textquoteright ve read related
to an issue of social justice. You\textquoteright ll also be asked
to respond to someone else\textquoteright s report. These assignments
will be completed online using Canvas discussions.

\section{Spinning the Numbers}
\begin{itemize}
\item tricks to spin the numbers {[}solicit from students{]}:
\begin{itemize}
\item Use of qualifiers (\textquotedbl not even 5\% of Americans\textquotedbl ,
\textquotedbl a whopping 80\% of schoolchildren\textquotedbl ) 
\item Use of numbers vs. percentages (\textquotedbl 1\% of Americans\textquotedbl{}
vs. \textquotedbl 3,000,000 Americans\textquotedbl ) 
\item Comparing to smaller or bigger numbers (\textquotedbl more Americans
die from lightning strikes than from terrorist attacks per year\textquotedbl ) 
\item Choice of summary statistics (mean, median, mode, range, percentages) 
\begin{itemize}
\item The mean is a measure of average that is skewed by outliers, while
the median is resistant to being skewed. 
\end{itemize}
\item Perform a study where you pick and choose a sample that will give
you the results you'd like 
\item Present cumulative data instead of quarterly data (or vice versa) 
\item Only look at data taken from a time range that fits your agenda 
\item Bad graphing: 
\begin{itemize}
\item Change the axis scales 
\item Remove scales on the axes (e.g. Planned Parenthood abortions vs. cancer
screenings chart) 
\item Frame the graph with a title or description that supports your agenda 
\item Use pictograms that allow you to skew the size of bars (e.g. textbooks,
onions, iPad batteries)
\end{itemize}
\end{itemize}
\end{itemize}
\begin{xca}
\textbf{(Worksheet) }Take each statistic below, and do the following: 
\begin{itemize}
\item Without looking it up, estimate what you think it might be. 
\item Now look it up online, looking for the most reliable estimate. 
\item If you wanted to make this number look big, how would you do that?
In other words, how would you contextualize the number so the reader
would think \textquotedblleft That\textquoteright s huge!\textquotedblright{} 
\item If you wanted to make the same number look small, how would you do
that? How would you contextualize the number so the reader would think,
\textquotedblleft That\textquoteright s not so big!\textquotedblright{}
\end{itemize}
\begin{enumerate}
\item The percentage of people of color among all people in Utah prisons
before criminal justice reforms in $2015$.
\begin{enumerate}
\item {[}slide{]} In $2014$, $34$\% of new prisoners were PoC
\end{enumerate}
\item {[}same slide{]} The percentage of people of color in Utah prisons
after criminal justice reforms in $2017$ (is there a problem here?)
\begin{enumerate}
\item In 2017, $43\%$ of new prisoners were PoC
\item {[}slides{]} mass incarceration starts with police interactions including
street checks and use of force
\item What factors other than racial discrimination may affect these numbers?
\item {[}slide{]} Recent Tribune article on allowing bias to be taken into
account in sentencing; plug bias talk 7-8:30pm in Jewett Center!
\end{enumerate}
\item Average American CO$_{2}$ emissions. 
\begin{enumerate}
\item Actual data: $19.8$ tons per person averaged over $1980-2006$
\item Small: compare total American emissions to China's total emissions
without discussing population differences
\item Large: the weight of $10$-$15$ cars
\end{enumerate}
\item The number of Dreamers (or, alternatively, the number of DACA recipients) 
\begin{enumerate}
\item How do you define Dreamer? Probably as an undocumented immigrant who
entered the US before their 18th birthday.
\item Guess: between $.1=1\times10^{-1}\%$ and $1=1\times10^{0}\%$ of
Americans. Take the geometric mean: $1\times10^{-0.5}$ or $\frac{1}{\sqrt{10}}\approx.32\%$. 
\item Compute: $.0032\times300\text{ million}\approx900,000$ Dreamers.
\item Small: Trump, Jan. 10: ``We want to see something happen with DACA.
It\textquoteright s been spoken of for years. And children are now
adults, in many cases. The numbers are very different, very varying.
A lot of people say 800,000; some people said \textemdash{} yesterday,
first time I heard 650 {[}thousand{]}. I also heard 3 million. The
fact is, our country was such a mess, nobody even knows what the numbers
are. But we\textquoteright ll know what the numbers are.''
\item 689,800 \textemdash{} This is the total number of people who had DACA
status as of Sept. 4, 2017, the day before the Trump administration
ended the program, according to USCIS. Nearly 80 percent of the DACA
recipients are from Mexico.
\item The number of undocumented immigrants who entered the U.S. before
their 18th birthday, the group known as DREAMers: $3.6$ million.
\item The number of undocumented immigrants who entered the U.S. before
their 16th birthday. This is the group that met the basic requirements
to apply for DACA: $1.8$ million
\item $800,000$: The total number of DREAMers who have received DACA protections
over the five years of the program.
\item small: $.2\%$ of the American population receives DACA protections
\item large: $1.2\%$ of the American population are DREAMers\textendash one
pesron on every Trax train(?)
\end{enumerate}
\item Total amount Americans will save with the recent Tax Reform Bill.
\begin{enumerate}
\item What is the average tax cut under the bill this year? Guess:
\begin{enumerate}
\item The average American's tax bill is around $\$10,000$ (from last time)
\item The average American probably receives between a $1\%$ and $10\%$
tax cut under the bill. The geometric mean is $1\times10^{0.5}\approx3\%$.
\item Compute: $3\%$ of $\$10,00$0 is $.03(10,000)=\$300$, our estimate
of the average tax cut.
\item Multiply our average by $300$ million Americans to get $\$90$ billion.
That sounds too huge!
\end{enumerate}
\item Actual data: overall average after-tax income increases by $1.7\%$
(\href{http://www.taxpolicycenter.org/sites/default/files/publication/148831/2001605-distributional-analysis-of-the-tax-cuts-and-jobs-act-as-passed-by-the-senate-finance-committee_1.pdf?wpisrc=nl_finance202&wpmm=1}{source})
or $\$1,300$. 
\begin{enumerate}
\item Thus $\$1,300$ per American times $300$ million Americans gives
$\$390,000,000,000$ saved. That's an order of magnitude higher than
our guess!
\end{enumerate}
\item Small: the average of $1.7\%$ is very skewed toward the rich, who
will be saving by far the most money under the bill.
\item Large: that's enough to pay rent on a 2br in SLC for a month!
\end{enumerate}
\item The number of incarcerated Americans
\begin{enumerate}
\item Actual number: $2,220,300$ or about $1\%$
\item Small: that's only one out of every $100$ Americans!
\item Large: every time you get on Trax there's one person who would have
ridden that train with you who is in prison instead.
\end{enumerate}
\item The percentage of Americans who support the Black Lives Matter movement. 
\begin{enumerate}
\item Guess: between $20\%$ and $40\%$. Geometric mean: $30\%$ of Americans.
\item Actual: $67\%$ of Americans surveyed by Pew in June 4-10, 2020, have
a positive view of Black Lives Matter:

\includegraphics[scale=0.5]{pasted6}
\item Small: about $30\%$ of Americans don't support Black Lives Matter!
\item Large: most Black Americans support Black Lives Matter, That $67\%$
is almost triple the percentage who supported the March on Washington
in $1963$.
\end{enumerate}
\item The percentage of Americans who supported the March on Washington
in August 1963, which aimed to draw attention to continuing challenges
and inequalities faced by African Americans a century after emancipation.
It was also the occasion of Martin Luther King, Jr.\textquoteright s
now-iconic \textquotedblleft I Have A Dream\textquotedblright{} speech.
\begin{enumerate}
\item Guess: between $40$ and $60$ percent. Geometric mean $\approx50\%$.
\item Actual: $23\%$ (\href{https://img.washingtonpost.com/wp-apps/imrs.php?src=https://img.washingtonpost.com/blogs/the-fix/files/2016/04/2300-galluppoll1963-1024x528.jpg&w=1484}{source})
\item Big: $1$ out of $4$ Americans is in favor of mass protest to promote
civil rights
\item Small: only $23\%$ of Americans are in favor of disruptive March
on Washington
\end{enumerate}
\end{enumerate}
\end{xca}


\section{Reading question: the MPG illusion}
\begin{xca}
Please read Sections 2.1-2.2 of~Common Sense Mathematics, on units
and MPG, then answer the following questions. Bring your answers to
class on the date below for discussion.
\begin{enumerate}
\item There are about 3.8 liters in a gallon and about 1.6 kilometers in
a mile. If a car can travel 30 miles for every gallon of gas it burns,
approximately how many kilometers can it travel per liter of gas?
Just like in Section 2.1 of~Common Sense Mathematics, write out your
units and cancel units just like you would cancel zeroes in a fraction. 
\begin{enumerate}
\item We use dimensional analysis:
\[
\frac{30\text{ mi}}{1\text{gal}}\times\frac{1\text{ gal}}{3.8\text{L}}\times\frac{1.6\text{ km}}{1\text{ mi}}=12.63\text{ km/L}
\]
\end{enumerate}
\item If a car can travel 30 miles per gallon of gas, how many gallons of
gas does the car need to travel one mile? Show all work and units,
cancelling units when needed. 
\begin{enumerate}
\item Here we note that, again just thinking about units (and ignoring the
numbers), that 
\[
\frac{mi}{gal}=1/(\frac{gal}{mi})
\]
so that if $x$ is the number of gallons of gas the car needs to travel
one mile, then
\[
x=1/\Big(\frac{30\text{ mi}}{1\text{ gal}}\Big)=\frac{1\text{ mi}}{30\text{ gal}}.
\]
\item This is really something we do whenever we use dimensional analysis:
in part (a), for example, we flipped the given $\frac{3.8L}{1\text{ gal}}$
and multiplied by $\frac{1\text{ gal}}{3.8L}$ in order to make the
units cancel out properly.
\end{enumerate}
\item If a car can travel 30 miles per gallon of gas, how many liters of
gas does the car need to travel one kilometer?~Show all work and units,
cancelling units when needed. 
\begin{enumerate}
\item We again use dimensional analysis, starting with $x$ from (b) and
using the dimensional analysis from (a):
\[
\frac{30\text{ mi}}{1\text{ gal}}=\frac{1\text{ gal}}{30\text{ mi}}\times\frac{3.8\text{ L}}{1\text{ gal}}\times\frac{1\text{ mi}}{1.6\text{ km}}=0.014\text{ L/km}.
\]
Note that, because we took the reciprocal of $30\text{ mpg}$ in order
to get gpm, we also have to take the reciprocal of everything we multiplied
by in (a) before multiplying here in (c).
\end{enumerate}
\item If you were looking to buy a new car, what, if anything, does reading
Section 2.2 change about what you'd consider before buying? Explain.
\begin{enumerate}
\item I'd convert everything to gpm and make my decisions based on that,
since gpm linearly correlates to fuel efficiency while mpg changes
in a harder-to-understand relationship to efficiency.
\end{enumerate}
\end{enumerate}
\end{xca}


\end{document}
