%% LyX 2.3.6.2 created this file.  For more info, see http://www.lyx.org/.
%% Do not edit unless you really know what you are doing.
\documentclass[oneside,english]{amsart}
\usepackage[T1]{fontenc}
\usepackage{geometry}
\geometry{verbose,tmargin=1in,bmargin=1in,lmargin=1in,rmargin=1in}
\usepackage{array}
\usepackage{amstext}
\usepackage{amsthm}
\usepackage{graphicx}

\makeatletter

%%%%%%%%%%%%%%%%%%%%%%%%%%%%%% LyX specific LaTeX commands.
%% Because html converters don't know tabularnewline
\providecommand{\tabularnewline}{\\}

%%%%%%%%%%%%%%%%%%%%%%%%%%%%%% Textclass specific LaTeX commands.
\numberwithin{equation}{section}
\numberwithin{figure}{section}
\theoremstyle{plain}
\newtheorem{thm}{\protect\theoremname}
\theoremstyle{definition}
\newtheorem{xca}[thm]{\protect\exercisename}
\theoremstyle{definition}
\newtheorem{defn}[thm]{\protect\definitionname}
\theoremstyle{definition}
\newtheorem{example}[thm]{\protect\examplename}
\theoremstyle{plain}
\newtheorem*{prop*}{\protect\propositionname}
\theoremstyle{plain}
\newtheorem{question}[thm]{\protect\questionname}

\makeatother

\usepackage{babel}
\providecommand{\definitionname}{Definition}
\providecommand{\examplename}{Example}
\providecommand{\exercisename}{Exercise}
\providecommand{\propositionname}{Proposition}
\providecommand{\questionname}{Question}
\providecommand{\theoremname}{Theorem}

\begin{document}
\title{WCSAM 12 - Diffie-Hellman Key Exchange}
\maketitle

\section*{Discussion: Key Exchange}

Put your chairs in a circle.
\begin{xca}
How do we solve the key distribution problem? In other words, suppose
Alice wants to send a message to Bob, and suppose they live in a country
in which any unencrypted message, or any key, will be intercepted
and read by the postal service. How do you make it so Alice doesn't
have to send Bob the key?
\end{xca}

\includegraphics[scale=0.75]{\string"Old Cryptography Notes/pasted192\string".png}

A potential solution?

\includegraphics[scale=0.75]{\string"Old Cryptography Notes/pasted191\string".png}

How do we implement this?

\includegraphics[scale=0.75]{\string"Old Cryptography Notes/pasted113\string".png}
\begin{xca}
\textbf{(Homework Discussion Question) }This analogy relies on encryption
being like putting a padlock on a box. Does this analogy hold up for
the ciphers we've used so far? The purpose of this activity is to
investigate that question. Please do the following:
\begin{enumerate}
\item Encrypt a short message of your choice using the affine cipher $C\equiv3P+1\mod26$
(call it \textquotedbl key A\textquotedbl ). Call the result of
this encryption \textquotedbl ciphertext A\textquotedbl . 
\item Encrypt ciphertext A using the affine cipher $C\equiv7P+6\mod26$.
This produces ciphertext that has been encrypted twice; call this
\textquotedbl ciphertext AB\textquotedbl .
\item Now try decrypting the ciphertext in two orders: first, decrypt using
key A, then decrypt using key B. Call this \textquotedbl plaintext
$ABA^{-1}B^{-1}$\textquotedbl . 
\item Now, try decrypting in the opposite order: first using key B, then
key A. Call the result \textquotedbl plaintext $ABB^{-1}A^{-1}$\textquotedbl . 
\item Which, if either, of these decryption methods produces clear plaintext?
Why do you think this is? Explain your answer.
\end{enumerate}
\begin{itemize}
\item The problem is that order matters for decryption, where it doesn't
matter for placing padlocks on a box. 
\item Encryption/decryption, for the most part, is like shoes and socks.
You must put on socks then shoes, and you must take off shoes then
socks. The other order doesn't work.
\item The formula for double-encryption is 
\[
C\equiv7(3P+1)+6\equiv21P+13\mod26
\]
\item The decryption equation for the first cipher is $P\equiv9(C-1)\mod26$
and the second decryption equation is $P\equiv-11(C-6)\mod26$. 
\item Thus, the formula for double-decryption is
\begin{align*}
P & \equiv9(-11(C-6)-1)\mod26\\
 & \equiv9(-11C+65)\\
 & \equiv9(-11C+13)\mod26\\
 & \equiv5C+13\mod26
\end{align*}
However, if we first decrypt cipher $A$, then cipher $B$, we'd get
\begin{align*}
P & \equiv-11[9(C-1)-6]\mod26.\\
 & \equiv-11(9C-15)\mod26.\\
 & \equiv5C+9\mod26
\end{align*}
which would not give plaintext. The output would be shifted back four
letters from the actual plaintext.
\item Decryption $ABB^{-1}A^{-1}$ produces clear plaintext, while $ABA^{-1}B^{-1}$
does not. Affine encryption is not \emph{commutative}!
\item \textbf{Challenge question: }which kind of affine ciphers \textbf{can
}be decrypted using $ABA^{-1}B^{-1}$? These are called the \textbf{linear}
ciphers. What is the difference between the graph of an affine vs.
a linear function?
\end{itemize}
\end{xca}

\begin{itemize}
\item What we really need is a mathematical process that works more like
padlocks than socks/shoes.
\item In other words, we want to be able to have Alice and Bob each add
their locks, then take them off in the opposite order in order to
decrypt the message.
\item And we want to do this without it being possible for anyone who intercepts
the message to decrypt it!
\item We want a \emph{trapdoor one-way function} (cipher) in which \emph{double-encryption
plus double-decryption, in any order, gives back the plaintext!}
\end{itemize}
\begin{defn}
Two mathematical functions or ciphers $f$ and $g$ are \textbf{commutative}
if doing first $f$, then $g$, gives the same result as doing first
$g$, then $f$:
\[
f(g(x))=g(f(x))
\]
no matter what $x$ is.
\end{defn}

\begin{xca}
What mathematical functions $f,g$ can you think of that are commutative
(doing function $1$ then function $2$ is the same as doing $2$
then $1$)? Which of the functions you thought of are two-way? One-way?
\begin{itemize}
\item \textbf{Commutative}: shift ciphers, affine ciphers with $k=0$, Vigenere
ciphers, Enigma, one-time pads
\begin{itemize}
\item $f(x)=x+C_{1}$, $g(x)=x+C_{2}$ (addition is commutative) 
\item $f(x)=m_{1}x$, $g(x)=m_{2}x$ (multiplication is commutative)
\item $f(x)=x^{a},g(x)=x^{b}$ (taking powers/exponents is commutative)
\end{itemize}
\item But all of the above are two-way since we can write formulas for undoing
them:
\begin{itemize}
\item $f^{-1}(x)=x-C_{1}$ (the opposite of addition is subtraction)
\item $f^{-1}(x)=\frac{1}{m_{1}}x$ (the opposite of multiplication is division)
\item $f^{-1}(x)=x^{1/a}$ (roots are the opposite of powers)
\end{itemize}
\end{itemize}
\end{xca}

\includegraphics{\string"Old Cryptography Notes/pasted118\string".png}
\begin{itemize}
\item Can you think of any other one-way functions that we've covered in
this class? 
\begin{itemize}
\item Multiplication mod $p$ - finding inverses is hard!
\end{itemize}
\end{itemize}
We'll actually encounter another one using modular arithmetic. Let's
learn about \textbf{modular exponentiation. }This turns out to be
the ``lock'' we use for the Diffie-Hellman key exchange.

\section*{Modular Exponentiation}
\begin{defn}
If $p$ is a positive whole number, then we define
\begin{align*}
b^{p} & =b\cdot b\cdots b\\
 & \text{ (\ensuremath{p} times)}.
\end{align*}
Similarly, $b^{p}\mod n$ is defined as the product $b\cdot b\cdots b\mod n$.
The process of reducing $b^{p}\mod n$ is called \textbf{modular exponentiation}.
\end{defn}

\begin{example}
Compute $25^{19}\mod113$.
\end{example}

\begin{itemize}
\item Exponents can get super large very quickly; for example, 
\[
25^{19}=363,797,880,709,171,295,166,015,625.
\]
\item If $p$ is large, then computing $b^{p}=25^{19}$ with the definition
above requires us to multiply $b=25$ by itself $p-1=18$ times (since
we start with one of the $b$s).
\end{itemize}
\begin{example}
However, we can be more efficient by considering the binary expansion
of $p$. For example, suppose we want to compute $25^{19}\mod113$.
The exponent $19$ in $25^{19}$ has binary expansion
\[
19=16+2+1=2^{4}+2^{1}+2^{0}=10011_{2}.
\]
\begin{prop*}
(Important) If $m$ and $n$are positive integers, than 
\[
b^{m+n}=b^{m}\times b^{n}.
\]
\end{prop*}
So we can write
\[
25^{19}=25^{16+2+1}=(25^{16})(25^{2})(25^{1}).
\]

Now, we can compute $25^{2}$ and $25^{16}$ by: 
\begin{enumerate}
\item repeatedly squaring $25$ until we hit the largest power of $25$,
i.e. $16$, reducing $\mod113$ at each step:
\begin{align*}
25^{2} & =625\equiv60\mod113\\
25^{4} & =(25^{2})^{2}\equiv60^{2}\equiv97\mod113\\
25^{8} & =(25^{4})^{2}\equiv97^{2}\equiv30\mod113\\
25^{16} & =(25^{8})^{2}\equiv30^{2}\equiv109\mod113.
\end{align*}
\item Multiplying the resulting powers of $25$ necessary to get $25^{19}$,
reducing after each multiplication:
\begin{align*}
25^{19} & =(25^{16})(25^{2})(25^{1})\\
 & \equiv109\times60\times25\mod113\\
 & \equiv-4\times60\times25\mod113\\
 & \equiv-100\times60\mod113\\
 & \equiv13\times60\mod113\\
 & \equiv102\mod113.
\end{align*}
Note that I multiplied $25\times-4$ before $60\times-4$ because
$25\times-4$ is close to $113$, so when I reduce it would become
a small number that's easier to multiply.
\end{enumerate}
\end{example}


\section{Square and multiply algorithm}
\begin{itemize}
\item This process works in general. To compute a large power of a number
mod $n$, say $b^{p}\mod n$, do the following:
\begin{enumerate}
\item Write $p$ as a sum of powers of $2$, for example using its binary
expansion, e.g. $12=2^{3}+2^{2}$ (since $12=1100_{2}$).
\item Use Step 1 and the Proposition on multiplication of bases equaling
addition of exponents to write $b^{p}$ as a product. For example,
since $12=8+4$, $3^{12}=3^{8}\times3^{4}$.
\item Square $b$ repeatedly, reducing at each step, until we reach the
largest power of $2$ from Step 2.
\begin{example}
\begin{align*}
3^{2} & =9\mod21\\
3^{4} & =9^{2}=81\equiv18\equiv-3\mod21\\
3^{8} & =(-3)^{2}=9\mod21.
\end{align*}
\end{example}

\item Multiply all the powers of $2$ from Step $3$ together, using the
computations from Step 4, reducing after each multiplication.
\[
3^{12}=3^{8}\times3^{4}\equiv9\times(-3)=-27\equiv-6\equiv15\mod21.
\]
\end{enumerate}
\item At the end of this process, called \textbf{the square and multiply
algorithm}, you will have computed this large power. 
\item Another way to speed up modular exponentiation is via the following
''important (proven) mathematical fact'' (or \textbf{theorem}):
\end{itemize}
\begin{thm}[Euler's Theorem]
 
\begin{enumerate}
\item If $a$ is any positive whole number and $p$ is a prime number so
that $\gcd(a,p)=1$, then $a^{p-1}\equiv1\mod p$.
\item If $p$ and $q$ are prime numbers and $a$ is any positive whole
number so that $\gcd(a,pq)=1$, then $a^{(p-1)(q-1)}\equiv1\mod pq$.
\end{enumerate}
\end{thm}

\begin{example}
To compute $9^{13}\mod14$, we first note that $14=7\times2$ and
that $2$ and $7$ are both prime. So by Euler's Theorem(2) and the
fact that $\gcd(9,14)=1$, we know that $9^{(7-1)(2-1)}=9^{6}\equiv1\mod14$. 

Now, we ``peel off'' two $9^{6}$s from $9^{13}$ by using the fact
that $9^{13}=9^{6}\times9^{6}\times9^{1}$:
\[
9^{13}\equiv9^{6}\times9^{6}\times9^{1}\equiv1\times1\times9^{1}=9^{1}=9\mod14.
\]
This saves us a ton of computation, especially for large powers, once
we get the hang of it!
\end{example}

\begin{xca}
Compute the following using the square and multiply algorithm and
Euler's Theorem.
\begin{enumerate}
\item $7^{52}\mod53$
\begin{enumerate}
\item Since $53$ is prime and $\gcd(7,53)=1$, $7^{52}\equiv1\mod53$.
\end{enumerate}
\item $7^{50}\mod51$
\begin{enumerate}
\item Since $51=3\times17$ and $\gcd(7,51)=1$, 
\[
7^{(17-1)(3-1)}=7^{16\times2}=7^{32}\equiv1\mod51
\]
and we can write
\begin{align*}
7^{50} & =7^{32}\times7^{28}\\
 & \equiv1\times7^{28}\mod51.
\end{align*}
\item Now, applying the square and multiply algorithm, we write 
\[
28=16+12=16+8+4=2^{4}+2^{3}+2^{2}
\]
so that 
\[
7^{50}\equiv7^{28}=7^{16}7^{8}7^{4}\mod51.
\]
\item Finally, we square $7$ repeatedly, reducing mod $51$ at each step:
\begin{align*}
7^{2} & =49\equiv-2\mod51\\
7^{4} & =(7^{2})^{2}=(-2)^{2}=4\mod51\\
7^{8} & =4^{2}=16\mod51\\
7^{16} & =16^{2}\equiv1\mod51
\end{align*}
hence
\[
7^{50}\equiv1\times16\times4=48\mod51.
\]
\end{enumerate}
\item $9^{100}\mod77$
\begin{enumerate}
\item We use the fact that $77=7\times11$ and $\gcd(9,77)=1$, so that
\[
9^{(11-1)(7-1)}=9^{10\times6}=9^{60}\equiv1\mod77.
\]
\item We can write 
\[
9^{100}=9^{60}9^{40}\equiv1\times9^{40}=9^{40}\mod77.
\]
\item We now use square and multiply, together with the fact that $40=32+8=2^{5}+2^{3}=101000_{2}$,
to compute:
\begin{align*}
9^{2} & =81\equiv4\mod77\\
9^{4} & =4^{2}=16\mod77\\
9^{8} & =16^{2}\equiv25\mod77\\
9^{16} & =25^{2}\equiv9\mod77\\
9^{32} & =9^{2}\equiv4\mod77
\end{align*}
hence
\[
9^{100}\equiv9^{40}=9^{32}9^{8}\equiv4\times25=100\equiv23\mod77.
\]
\end{enumerate}
\end{enumerate}
\end{xca}

\includegraphics[scale=0.75]{\string"Old Cryptography Notes/pasted156\string".png}
\begin{question}
With the square and multiply algorithm, it's not that hard to compute
powers in modular arithmetic! Is computing powers this way a one-way
or a two-way function?
\end{question}

\begin{itemize}
\item In regular arithmetic, computing powers is a two-way function: going
forward, we can compute $3^{4}=81$ with a calculator. 
\item Going backward, we could solve $3^{x}=81$ by making guesses: $5$
is too big because $3^{5}=243$, so we'd check $4$ and be correct.
We could also just plug $\log_{3}81$ into our calculator to get $x=4$!
\item In modular arithmetic, this is a one-way function. Suppose that we
are told that $3^{x}\equiv1\mod7$ and we guess $x=5$. We'd compute
$3^{5}\mod7$ to be $5$, which is too big, because we are looking
for an answer of just $1$. We might try reducing the value of $x$
and trying again. But we'd be heading in the wrong direction, because
the actual answer is $x=6$.
\end{itemize}
\begin{xca}
Fill out a table by calculating the values of $3^{x}$ mod $7$ for
different $x$ values.
\end{xca}

See how erratic this is? This is a classic example of a one-way function.
\begin{itemize}
\item In other words, above we computed that $25^{19}\equiv83$ mod $103$.
Let's say we're given $83$ mod $103$ and want to know what power
of $25$ equals $83$. How would you do this?
\item Really hard, right? You'd have to do trial and error. That's the basis
of the Diffie-Hellman key exchange.
\item So now we have a one-way function! That's good because a one-way function
is like a padlock: easy to put on, hard to take off.
\item Now Alice has padlocked her box and sent it to Bob, and Bob has padlocked
his and sent it to Alice. How are they going to \textbf{unlock} the
boxes without Alice having to send her key to Bob or Bob sending his
key to Alice?
\end{itemize}

\section*{The Diffie-Hellman Algorithm}
\begin{xca}
\textbf{(reading question) {[}slide{]}}
\end{xca}

\begin{enumerate}
\item In the color example, what sequence of color mixes does Alice perform?
What sequence of color mixes does Bob perform? 
\begin{enumerate}
\item Alice mixes her private red with the public yellow, then receives
Bob's mixture and adds her private red.
\item Bob mixes his private blue with the public yellow, then receives Alice's
mixture and adds his private blue.
\end{enumerate}
\item How is it possible that both of these sequences arrive at the same
shared color? 
\begin{enumerate}
\item Both sequences mixed yellow + red + blue.
\end{enumerate}
\item How is it that the colors Alice and Bob send to each other don't reveal
either of their private colors nor the shared color? 
\begin{enumerate}
\item The shared and public colors are both obscured through the process
of mixture, which is a one-way function that it's very difficult to
undo (unmix).
\end{enumerate}
\item Fill in the second column of the following table to explore the connections
between the color analogy and the Diffie-Hellman key exchange (which
uses modular exponentiation): 

\begin{tabular*}{5in}{@{\extracolsep{\fill}}|>{\centering}p{2in}||>{\centering}p{2in}|}
\hline 
\textbf{Color analogy} & \textbf{Diffie-Hellman key exchange}\tabularnewline
\hline 
\hline 
Mixing colors is easy, but unmixing them is hard & Modular exponentiation is easy, but the discrete logarithm is hard\tabularnewline
\hline 
\hline 
Public, shared yellow & Public, shared base\tabularnewline
\hline 
\hline 
Alice's private red & Alice's private exponent\tabularnewline
\hline 
\hline 
Bob's private blue & Bob's private exponent\tabularnewline
\hline 
\hline 
The mixtures of public and private colors that are sent publicly & The shared base to the power of the private exponent\tabularnewline
\hline 
\hline 
The shared private color that Alice and Bob agree on & The shared result of taking the shared base to \textbf{both }powers\tabularnewline
\hline 
\end{tabular*}
\end{enumerate}
Let's break down what's going on here.
\begin{example}
Call people up to be Alice, Bob, and Eve. Alice is trying to send
a secret message to Bob. Eve is trying to Eves-drop (geddit?). {[}Analogize
to colors at each stage.{]}
\end{example}

\includegraphics[scale=0.75]{\string"Old Cryptography Notes/pasted122\string".png}
\begin{itemize}
\item Just as mixing red then blue into yellow is the same as mixing blue
then red, computing $(3^{A})^{B}$ is the same as computing $(3^{B})^{A}$!
Both color mixing and modular exponentiation work like padlocks (order
doesn't matter), not like socks/shoes (order does matter).
\end{itemize}
\begin{xca}
Perform the key exchange in pairs with the shared base $5$ and the
modulus $7$.
\end{xca}

\includegraphics[scale=0.75]{\string"Old Cryptography Notes/pasted166\string".png}

\includegraphics{\string"Old Cryptography Notes/pasted167\string".png}
\begin{example}
{[}slide{]} Using computers, we can carry out this process for huge
prime numbers and still agree on a shared secret key!
\end{example}

\includegraphics[scale=0.75]{\string"Old Cryptography Notes/pasted164\string".png}

\textbf{Journal Entry.} Out of the cryptological breakthroughs we've
discussed so far, which do you think is most important? Why?
\end{document}
