%% LyX 2.3.6.2 created this file.  For more info, see http://www.lyx.org/.
%% Do not edit unless you really know what you are doing.
\documentclass[oneside,english]{amsart}
\usepackage[T1]{fontenc}
\usepackage{geometry}
\geometry{verbose,tmargin=1in,bmargin=1in,lmargin=1in,rmargin=1in}
\usepackage{amstext}
\usepackage{amsthm}

\makeatletter
%%%%%%%%%%%%%%%%%%%%%%%%%%%%%% Textclass specific LaTeX commands.
\numberwithin{equation}{section}
\numberwithin{figure}{section}
\theoremstyle{plain}
\newtheorem{thm}{\protect\theoremname}
\theoremstyle{definition}
\newtheorem{xca}[thm]{\protect\exercisename}
\theoremstyle{plain}
\newtheorem{question}[thm]{\protect\questionname}

\makeatother

\usepackage{babel}
\providecommand{\exercisename}{Exercise}
\providecommand{\questionname}{Question}
\providecommand{\theoremname}{Theorem}

\begin{document}
\title{WCSBS 220 - Fermi Estimation}
\maketitle

\subsection{}

\section{Discussion - Bob Moses \& Math Literacy}
\begin{xca}
(TPS; \textbf{Reading Question}) Bob Moses is a veteran of the Civil
Rights Movement and a former leader of the Student Nonviolent Coordinating
Committee (SNCC), a major Civil Rights Movement organization dedicated
to community organizing. He is also a math educator and founder of
the Algebra Project, which works to promote mathematical literacy
among low-income students and students of color. Please read the attached
chapter~from Moses' book, Radical Equations, entitled \textquotedbl Algebra
and Civil Rights?\textquotedbl{} While reading, think about your sense
of the connections between mathematics and civil rights, equity, fairness,
and justice. Then answer the questions below.
\begin{enumerate}
\item What is Ella Baker's definition of \textquotedbl radical\textquotedbl{}
in the quote at the beginning of the chapter? How is Baker's definition
similar to or different from your own? 
\begin{enumerate}
\item ``This means that we are going to have to learn to think in radical
terms. I use the term radical in its original meaning\textemdash getting
down to and understanding the root cause. It means facing a system
that does not lend itself to your needs and devising means by which
you change that system.''
\item rad=root (e.g. raiz in Spanish)
\item students' definitions:
\begin{enumerate}
\item Melodey: bringing change through extremist ways
\end{enumerate}
\end{enumerate}
\item In your own words, why does Moses believe that \textquotedbl the
most urgent social issue affecting poor people and people of color
is economic access\textquotedbl{} and that \textquotedbl economic
access and full citizenship depend crucially on math and science literacy\textquotedbl{}
(p. 5)? 
\begin{enumerate}
\item According to Moses, economic access is needed to lift marginalized
and inner-city residents out of violence and criminalization.
\item Moses states that the need for ``knowledge workers'' is a primary
trend of the contemporary economy, earning $82\%$ more than workers
in other industries.
\item Moses argues that ``illiteracy in math is acceptable the way illiteracy
in reading and writing is unacceptable'', causing a socially-constructed
lack of fluency in math.
\item ``Math illiteracy is not unique to Blacks the way the denial of the
right to vote in Mississippi was. But it affects Blacks and other
minorities much, much more intensely, making them the designated serfs
of the information age just as the people that we worked with in the
s on the plantations were Mississippi\textquoteright s serfs then.''
\item ``How do the people at the bottom get into the mix?''
\begin{enumerate}
\item Why the underlying assumption that marginalized groups should ``get
into the mix'' instead of changing the structure of the mix, getting
rid of the mix entirely and starting something new?
\end{enumerate}
\end{enumerate}
\item How do you react to Moses' assertion that \textquotedbl not being
'good' in math does not in any way imply inferiority, rather, it confirms
that you're just like most everyone else\textquotedbl{} (p. 10)? How
does this assertion connect to growth mindset and productive failure?
\begin{enumerate}
\item address the fallacy that not trying ``hard enough'' at math means
setting yourself up for failure
\item Maddie: general understanding that ``it's ok to not understand math...you'll
never need it''
\item growth mindset is not always realistic
\end{enumerate}
\item How have the issues Moses discussed changed since his book was published
in early 2002? How have they stayed the same? 
\begin{enumerate}
\item lots of Black folks are still poor
\item residential segregation still happens today
\item a lot of what happened is still happening, but it's more implicit
\end{enumerate}
\item In what ways is Moses' approach to the Algebra Project similar to
and different from his approach to the civil rights movement? Explain. 
\begin{enumerate}
\item similar: Algebra Project as deserving equal time as civil rights movement
\item difference: block-walking is less emphasized now, there was no Internet
during initial civil rights movement
\end{enumerate}
\item In your own words, is learning mathematics important for civil rights,
fairness, and social justice? Why or why not?
\begin{enumerate}
\item Ella Baker's ``radical'' definition: math is helpful in understanding
the root cause, increases understanding of the issue through numbers
\item Evidence/data for disparities
\item Overemphasis on data? Data as ``backup'' while ignoring personal
experience
\end{enumerate}
\end{enumerate}
\end{xca}

\begin{itemize}
\item For class discussion: 
\begin{itemize}
\item Other than mathematical literacy, what other important civil rights
and social justice issues impact low-income people and people of color
in contemporary society? How might we use math to address those issues?
We'll try to find ways.
\item What groups other than low-income folks and people of color are being
underserved in math literacy education? Explain.
\item Is algebra still the gatekeeper of the math curriculum?
\begin{itemize}
\item Importance of tutoring and Westminster's tutoring collaborations
\end{itemize}
\end{itemize}
\end{itemize}

\section{Reading the News Week 1}
\begin{itemize}
\item Post links to your articles please!
\item Many people looked up statistics and just compared the statistics
they found online to each other. This is valid, but the goal of this
and other assignments on estimation is to be able to practice estimating
more complex or unavailable data quickly, on the ``back of an envelope''.
I went easy this time, but on our in-class work today, try not to
use the Internet to look up statistics.
\end{itemize}

\section{Top vote-getters in discussion poll}
\begin{enumerate}
\item Class \& income inequality
\item Four-way tie: 
\begin{enumerate}
\item 3) human trafficking
\item environmental justice
\item how math perpetuates inequality
\item 1) racial justice
\item 2) LGBTQ+ issues
\end{enumerate}
\end{enumerate}
\begin{xca}
Take a ranked-choice class vote between these four. \textbf{Voters
pick a first-choice candidate and have the option to rank backup candidates
in order of their choice: second, third, and so on. If a candidate
receives more than half of the first choices, that candidate wins,
just like in any other election. However, if there is no majority
winner after counting first choices, the race is decided by an \textquotedbl instant
runoff.\textquotedbl{} The candidate with the fewest votes is eliminated,
and voters who picked that candidate as \textquoteleft number 1\textquoteright{}
will have their votes count for their next choice. This process continues
until a candidate wins with more than half of the votes. }
\end{xca}


\section{News update: fivethirtyeight election forecasts}
\begin{xca}
how do these forecasts estimate the likelihood of Biden winning the
2020 election? How much do we trust them? We'll dig into their (simulation-like)
methodology later in the course and find out!

In the wake of the mass shooting last Saturday, August 31, in Odessa,
TX, mass shootings are once again in the national consciousness. (for
how long?) Read over the poll at https://tinyurl.com/HuffPollGuns
and answer the following questions:
\begin{enumerate}
\item What surprises you about the results to this poll?
\item This poll was taken between August 5-6, 2019. What happened near that
date that might have affected the results of the poll?
\begin{enumerate}
\item August 3: El Paso anti-immigrant mass shooting
\item August 4: Dayton, OH, mass shooting killing 9
\end{enumerate}
\item This was a poll of $1000$ people. If you were conducting the poll,
how would you determine whether these $1000$ people's views matched
those of the whole United States? If you were choosing the $1000$
people, how would you choose them?
\begin{enumerate}
\item Throw darts at a phone book? Great!
\item Find people just hanging out outside? 
\end{enumerate}
\end{enumerate}
\end{xca}

~
\begin{xca}
{[}RQ: Is math everywhere?{]} 
\begin{enumerate}
\item What arguments does Barany give against the statement \textquotedbl math
is everywhere\textquotedbl ? How would you reply to his arguments? 
\item What arguments does Haensch give that \textquotedbl math is still
everywhere\textquotedbl ? How would you reply to her arguments? 
\item After reading both articles, what perspective do you have on the way(s)
mathematics interacts with society, power, and oppression?
\item Haensch argues that Barany's arguments do not support the title of
his blog post. Do you agree? Why or why not? If you agree, how would
you retitle Barany's piece?
\end{enumerate}
\end{xca}

~
\begin{xca}
{[}Reading Question: CSM Exercise 1.8.2{]} In his May 17, 2010 op-ed
column in The New York Times Bob Herbert noted that the dropout rate
for American high school students was one every 26 seconds. {[}R10{]}
Is this number reasonable?
\begin{itemize}
\item Let's estimate. 
\item First, how many American high school students are there? We'll assume
this corresponds with the proportion of Americans between ages $14$
and $18$, inclusive. This is probably between $5$ and $15\%$ of
Americans.
\item We use a geometric mean to get a single estimate:
\begin{align*}
5\% & =5\times10^{0}\%\\
15\% & =1.5\times10^{1}\%\\
\text{geometric mean} & =\frac{1.5+5}{2}\times10^{0.5}=3.25\times10^{0.5}\approx10\%.
\end{align*}
\item around $10\%$ (guess) of Americans, and there are about $300$ million
Americans, so we estimate that there are 
\[
.1\times300,000,000=30,000,000
\]
American high-school students each year. Our answer here is ``tens
of millions'', which fits with the actual number of about $56.6$
million. {[}Note that we don't round; we just count the number of
zeroes, which here is $7$, meaning our answer is ``tens of millions''.{]}
\item Assume that $20\%$ (perhaps a high estimate) of American high-school
students drop out each year. That's
\[
.2(30,000,000)=6,000,000
\]
dropouts per year.
\item We convert to dropouts per second:
\[
\frac{6,000,000\text{ dropouts}}{\text{year}}\times\frac{1\text{ year}}{365\text{ days}}\times\frac{1\text{ day}}{24\text{ hours}}\times\frac{1\text{ hour}}{60\text{ min}}\times\frac{1\text{ min}}{60\text{ sec}}=0.19\approx\frac{0.2\text{ dropouts}}{\text{second}}.
\]
\item Finally, we compute
\[
\frac{0.2\text{ dropouts}}{\text{second}}\times\frac{26\text{ seconds}}{1\text{ 26-second period}}=\frac{5.2\text{ dropouts}}{\text{26-second period}}
\]
which is higher (rounded up) than Herbert's estimate. However, we
note that we purposefully overestimated the dropout rate in order
to be generous to Herbert's estimate, so we could easily be off by
a factor of $5$ ($1$ vs. $5.2$ dropouts per $26$ seconds).
\end{itemize}
\end{xca}


\section{Fermi Estimation}
\begin{itemize}
\item In order to verify claims made in the media, to compute how much sea
level rise would be associated with a $1^{\circ}$ rise in global
sea temperature, or to test whether the claims Trump makes in his
next State of the Union address are correct, it's essential to have
an ability to estimate\textendash to lie (or oversimplify) skillfully
and in a way that lets us better understand the truth.
\item I make many mistakes on this type of problem until I think it through
carefully!{]}
\item Tools for ``back-of-the-envelope calculations'' aka Fermi estimation
\begin{itemize}
\item \textbf{Estimation}\textendash round up or down to make calculations
easier and numbers less messy. As we saw when estimating the number
of classrooms needed for the US household wealth, rounding before
the computation ends can introduce error. Therefore, if you have a
calculator, it's best to plug into the calculator before rounding,
whereas if you're computing by hand, rounding may be necessary in
order to increase the accuracy of your computations. Everyone makes
arithmetic mistakes\textendash you'll see me do it plenty of times!
\item \textbf{Dimensional analysis}: start with the given and multiply by
fractions, canceling out the units, until you get something in the
units your answer should be in. This is almost always your answer!
\item \textbf{Only guess up to an order of magnitude}, i.e. the number of
zeroes.
\end{itemize}
\item Why bother with Fermi estimates, if your estimates are likely to be
off by a factor of 2 or even 10? Often, getting an estimate within
a factor of 10 or 20 is enough to make a decision. So Fermi estimates
can save you a lot of time, especially as you gain more practice at
making them.
\end{itemize}
\begin{question}
Can we say for sure that there were less than $300$ million internatinoal
phone calls per day in $2007$?
\begin{itemize}
\item Nope! We only can be accurate up to, at most, an \emph{order of magnitude}\textendash that
is, there were probably millions but not billions of international
phone calls made by Americans per day in $2007$.
\item We can say our \emph{estimate }is $3\times10^{8}$ or $300,000,000$,
but the only part we're fairly sure about is the eight zeroes, the
$10^{8}$.
\end{itemize}
\end{question}


\section{Reading Questions: WMD Intro}
\begin{xca}
Please read the Introduction of Weapons of Math Destruction, by the
data scientist Cathy O'Neil (Links to an external site.)Links to an
external site., then give the following questions your best attempt.
Bring your answers to class for discussion.
\begin{enumerate}
\item According to O'Neil, what is a \textquotedbl weapon of math destruction\textquotedbl ?
\begin{enumerate}
\item a model that manages our lives and tends to reproduce social inequities
or inherit prejudice from its programmers/source datasets
\end{enumerate}
\item What \textquotedbl paradox\textquotedbl{} does O'Neil note about
the standard of evidence WMDs are held to as opposed to humans? How
might this paradox lead to social injustice?
\begin{enumerate}
\item They're black-boxed and need provide less evidence to justify profiling
a person than that person must provide to exonerate themselves
\end{enumerate}
\item According to O'Neil, in what ways do WMDs differ from the \textquotedbl effective
use{[}s{]} of statistics\textquotedbl{} in Google ad testing?
\begin{enumerate}
\item not enough controls/variables to monitor
\item no feedback from testing certain variables to conclude that they don't
affect the outcome
\item ``the privileged...are processed more by people, the masses by machines''
\end{enumerate}
\item What \textquotedbl payoff\textquotedbl{} do the companies who make
WMDs receive for their effort? According to O'Neil, how does this
\textquotedbl payoff\textquotedbl{} lead to a \textquotedbl dangerous
confusion\textquotedbl ?
\begin{enumerate}
\item they get lots of money and think it's because they're on the right
track
\item they lose track of the people ``on the receiving end of the transaction''
\end{enumerate}
\item What question(s) do you have about data science, statistical models,
and WMDs after reading the introduction to O'Neil's book?
\end{enumerate}
\end{xca}

\begin{itemize}
\item Tips for estimation:
\begin{itemize}
\item Change the rules to make arithmetic easier. There weren't exactly
$300$ million Americans in $2007$\textemdash the actual population
was something like $301.2$ million. But since we're just giving an
estimate anyway, we don't care about the $1.2$. We \textbf{round}
to the nearest hundred million!
\begin{itemize}
\item For this reason, we only care about how many zeroes your answer has,
not the actual number. So for our purposes \textbf{$1,000,000$ and
$9,000,000$ are the same }because they give us the answer of ``millions''.
\end{itemize}
\item Writing all those zeroes can get unwieldy, so we use \textbf{scientific
notation}: $1$ with $14$ zeroes after it is the same as $1\times10^{14}$.
Nine with seventy zeroes after it is written $9\times10^{70}$.
\item \textbf{Divide and conquer}: split the problem up by first figuring
out all the things we need to know, then using those things in some
order to get an answer.
\item \textbf{Use ranges. }If you feel fairly confident that there are between
$1000$ and $300000$ Americans who don't like Harry Potter (between
$1\times10^{3}$ and $3\times10^{5}$ in scientific notation), average
the exponents and the single-digit numbers to get your estimate. Here
we'd average $1$ and $3$ to get $2$ and $3$ and $5$ to get $4$,
so your answer would be $2\times10^{4}=20,000$ Americans.
\begin{itemize}
\item This is called taking a \textbf{geometric mean}, which is easier to
take than a ``regular'' average when dealing with big numbers and
exponents.
\end{itemize}
\item \textbf{Look up related data}, if it's available, to check the reasonableness
of your estimation. For example, when estimating the number of active
COVID-19 cases in SLC, you could look up scientific estimates of the
proportion of COVID-19 cases that are reported, multiply that proportion
by the population of SLC, and see if it's in the same ballpark (same
number of zeroes) as your estimate.
\end{itemize}
\end{itemize}
\begin{xca}
(Worksheet on Canvas) Estimation. \textbf{No computers/looking up
allowed bc the goal is to practice a skill that is useful when we
don't have real-world data! }Divide the students into teams of 3-4.
Each team starts on one of the first seven estimates with instructions
to move on to the next one when done, circling back to Google search
if they reach the end of the list. The teams start in different places,
so after about half an hour the class has found two or three estimates
for each of the seven tasks. The different answers for each task should
have the same order of magnitude. If they don\textquoteright t, we
try to figure out why not. (Good to have poster paper or whiteboards
for this)
\begin{enumerate}
\item Annual income for $1$ full-time Utah minimum wage worker.
\begin{enumerate}
\item Federal (and Utah) minimum wage is $\$7.25$/hour. Round to $\$10$
to make the math easier (we only care about the number of zeroes in
our answer, so this should turn out fine).
\item A full-time worker works $40$ hours/week.
\item There are $52\approx50$ weeks in a year. 
\item Thus the annual income is approximately
\[
\frac{\$10}{\text{hour}}\times\frac{40\text{ hours}}{\text{week}}\times\frac{50\text{ weeks}}{\text{year}}=\frac{\$20,000}{\text{year}}.
\]
It's in the tens of thousands of dollars. Notice how all the units
cancel out and the answer is in dollars per year, as desired, while
our given information was in dollars per hour. At each step, we're
converting from one unit to another (dollars/hour to dollars/week
to dollars/year).
\item Real answer (for federal/Utah minimum wage): 
\[
\frac{\$7.25}{\text{hour}}\times\frac{2,080\text{ hours}}{\text{year}}=\$15,080
\]
also in the tens of thousands! Note how close our answer is. This
is because our rounding (up and down) canceled itself out.
\end{enumerate}
\item Average Westminster student\textquoteright s household (family) income.
\begin{enumerate}
\item Mean or median?
\item Can we agree that the answer should be higher than $\$100,000$? 
\item Should it be a whole order of magnitude (ten times) higher? That would
make the average Westminster student's family income $\$1,000,000$
or above. This is probably a bit overgenerous, so our answer should
be $\$100,000$.
\item https://tinyurl.com/WMIncome: real median household income (according
to anonymous tax records; is this representative?) is $\$103,900$;
we were very close!
\end{enumerate}
\item Average University of Utah student's household (family) income 
\begin{enumerate}
\item Can we assume that the average U student's family income is less than
$80\%$ of a Westminster student's family income? That would put it
around $0.8(\$100,000)=\$80,000$, so in the tens of thousands of
dollars.
\end{enumerate}
\item Number of active COVID-19 cases in Salt Lake City. (Not just confirmed
cases.) 
\begin{enumerate}
\item By some estimates, only $10\%$ of COVID cases are reported.
\begin{question}
How could we verify this using a scientific study?
\end{question}

\item According to an article published in \emph{JAMA Internal Medicine},
''From comparing the number of estimated cases to reported case figures
on the last day of sample collection, the study\textquoteright s authors
estimated that there were six times as many COVID-19 infections in
Connecticut, the lowest figure, and potentially 24 times the number
of infections in Missouri than were reported.''
\begin{enumerate}
\item This cross-sectional study performed serologic testing on a \textbf{convenience
sample} of residual sera obtained from persons of all ages. 
\end{enumerate}
\end{enumerate}
\item Number of active COVID-19 cases among Westminster students. (Not just
confirmed cases.) 
\item Percentage of active COVID-19 cases in Salt Lake City that occur in
people of color.
\begin{enumerate}
\item According to the CDC, ''Non-Hispanic American Indian or Alaska Native
persons and non-Hispanic Black persons had age-adjusted hospitalization
rates approximately 4.7 times that of non-Hispanic White persons.
The rate for Hispanic or Latino persons was approximately 4.6 times
the rate among non-Hispanic White persons.''
\end{enumerate}
\item Average tax rate for an American in the wealthiest $1\%$.
\begin{enumerate}
\item We use the given facts; first, we note the average American in the
wealthiest $1\%$ earns $\$2.5$ million per year, $40\%$ of which
($\$1$ million) is made in capital gains. This means $\$1.5$ million
is not made from capital gains and is taxed at the higher $40\%$
tax rate.
\item The tax on the capital gains portion of income is easier to estimate:
\[
0.2(\$1,000,000)=\$200,000
\]
\item The tax rate on the non-capital gains income would be $\$150,000$
plus $40\%$ of the amount over $\$500,000$. Since the non-capital
gains income totals $\$1,500,000$, the amount over $\$500,000$ is
$\$1$ million. So the tax on the non-capital gains income would be
\[
\$150,000+0.4(\$1,000,000)=\$150,000+\$400,000=\$550,000.
\]
\item Thus, the total tax paid by the average member of the $1\%$ is $\$550,000+\$200,000$.
To find the tax \emph{rate} paid, we have to figure out what percent
of their income this average $1$-percenter paid. They paid $\$750,000$
in taxes, and their income was $\$2.5$ million, so they paid a tax
rate of
\[
\frac{\$750,000}{\$2,500,000}\times100\%=30\%.
\]
\end{enumerate}
\item Average tax rate for an American teacher.
\begin{enumerate}
\item The average teacher earns $\$60,000$ a year. This means their tax
rate is $\$14,000$ plus $25\%$ of the amount over $\$80,000$. Since
there's no amount over $\$80,000$, this teacher pays $\$14,000\approx\$15,000$
in taxes.
\item $\$15,000$ is $\frac{1}{4}$ or $25\%$ of the teacher's income of
$\$60,000$. 
\end{enumerate}
\item Is Joe Biden's statement reasonable?
\begin{enumerate}
\item The teacher's tax rate looks at first glance to be smaller than the
top $1\%$er's tax rate, but note we estimated. Both of our answers
are in ''tens of thousands'', so we can't 
\item Due to potential estimation error, it's actually reasonable that Biden's
statement is true.
\end{enumerate}
\item Average number of Americans who die each day due to mass shootings.
\begin{enumerate}
\item How to define a ``mass shooting''?
\item No official definition: at least $4$ killed? $4$ injured? Including
shooter or not? Including violence due to drugs/gangs?
\item Estimate the number of mass shootings in $2018$: between $10=1\times10^{1}$
and $100=1\times10^{2}$. Take the geometric mean: $1\times10^{1.5}\approx30$.
\item Estimate the number of Americans killed in each mass shooting: say
$5$ on average. So a total of 
\[
5\times30=150
\]
Americans are estimated to have been killed by mass shootings in $2018$.
\end{enumerate}
\item Number of Americans killed by undocumented immigrants in $2018$.
\begin{enumerate}
\item Note that no data exists.
\item Given that there were $15,000$ murders total, we need to estimate
the number of undocumented immigrants in the US in $2018$, as a percentage
of the population. Say between $1$ and $10\%$, with geometric mean
$3\%$. Very close to the real number!
\item You may have heard the (true) claim that undocumented Americans commit
crimes at a lower rate than documented Americans. 
\item Even if you assume the rate is the same, that means that $3\%$ of
the $15,000$ murders, or 
\[
.03(15000)=450\approx500
\]
murders were committed by undocumented immigrants in $2018$.
\item Both the mass shooting and murder numbers are in the hundreds, so
there's absolutely no evidence for the ``tens of thousands'' claim
made in the post.
\end{enumerate}
\item \# of Americans arrested each second for possession of weed alone
from 2001-2010.
\begin{enumerate}
\item We know about $600,000$ Americans were arrested for weed alone in
$2010$. Assuming weed policies were more lenient in $2010$ than
in $2001$, we'd guess maybe an average of at most $1,000,000$ Americans
a year over this $10$-year period, for a total of 
\[
10\times1,000,000=10,000,000
\]
Americans arrested.
\item We now estimate seconds/year:
\[
1\text{ year}\times\frac{350\text{ days}}{\text{year}}\times\frac{25\text{ hours}}{\text{day}}\times\frac{60\text{ min}}{\text{hr}}\times\frac{60\text{ sec}}{\text{min}}=30,000,000\text{ sec}.
\]
\item This means that there were 
\[
\frac{10,000,000}{30,000,000}\approx0.3\text{ arrests/sec}
\]
or about one arrest every $3$ sec. 
\item Now note that multiple people could be swept up in a single pot bust.
Assume $2$ people on average per bust, that gives
\[
0.3\frac{\text{arrests}}{\text{sec}}\times\frac{1\text{ bust}}{2\text{ arrests}}=0.15\frac{\text{busts}}{\text{sec}}
\]
or $15$ busts every $100$ sec. This is on the same order of magnitude
as the ACLU's estimate, so their estimate is reasonable.
\end{enumerate}
\end{enumerate}
\end{xca}

~
\begin{xca}
{[}or next time{]} Please read Sections 2.1-2.2 of Common Sense Mathematics,
on units and MPG, then answer the following questions. Bring your
answers to class on the date below for discussion.
\begin{enumerate}
\item There are about 3.8 liters in a gallon and about 1.6 kilometers in
a mile. If a car can travel 30 miles for every gallon of gas it burns,
approximately how many kilometers can it travel per liter of gas?
Just like in Section 2.1 of Common Sense Mathematics, write out your
units and cancel units just like you would cancel zeroes in a fraction.
Hint: start with $30\frac{\text{miles}}{\text{gallon}}$. Then multiply
by a number with units $\frac{\text{kilometers}}{\text{miles}}$ in
order to cancel the \textquotedbl miles\textquotedbl{} and replace
it with \textquotedbl kilometers\textquotedbl . 
\item If a car can travel 30 miles per gallon of gas, how many gallons of
gas does the car need to travel one mile? Show all work and units,
cancelling units when needed. 
\item If a car can travel 30 miles per gallon of gas, how many liters of
gas does the car need to travel one kilometer? Show all work and units,
cancelling units when needed. 
\item If you were looking to buy a new car, what, if anything, does reading
Section 2.2 change about what you'd consider before buying? Explain.
\end{enumerate}
\end{xca}


\end{document}
