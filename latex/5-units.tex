%% LyX 2.3.6.2 created this file.  For more info, see http://www.lyx.org/.
%% Do not edit unless you really know what you are doing.
\documentclass[oneside,english]{amsart}
\usepackage[T1]{fontenc}
\usepackage{geometry}
\geometry{verbose,tmargin=1in,bmargin=1in,lmargin=1in,rmargin=1in}
\usepackage{amstext}
\usepackage{amsthm}
\usepackage{graphicx}

\makeatletter
%%%%%%%%%%%%%%%%%%%%%%%%%%%%%% Textclass specific LaTeX commands.
\numberwithin{equation}{section}
\numberwithin{figure}{section}
\theoremstyle{plain}
\newtheorem{thm}{\protect\theoremname}
\theoremstyle{definition}
\newtheorem{xca}[thm]{\protect\exercisename}

\makeatother

\usepackage{babel}
\providecommand{\exercisename}{Exercise}
\providecommand{\theoremname}{Theorem}

\begin{document}
\title{WCSBS 220 4 - Units and Unit Conversions}
\maketitle

\section{Reading question: the MPG illusion}
\begin{xca}
Please read Sections 2.1-2.2 of~Common Sense Mathematics, on units
and MPG, then answer the following questions. Bring your answers to
class on the date below for discussion.
\begin{enumerate}
\item There are about 3.8 liters in a gallon and about 1.6 kilometers in
a mile. If a car can travel 30 miles for every gallon of gas it burns,
approximately how many kilometers can it travel per liter of gas?
Just like in Section 2.1 of~Common Sense Mathematics, write out your
units and cancel units just like you would cancel numerators and denominators
in a fraction. 
\begin{enumerate}
\item We convert mi/gal (mpg) to gpm (gal/mi) by taking a reciprocal:
\[
30\frac{\text{mi}}{\text{gal}}=\frac{1\text{ gal}}{30\text{ mi}}
\]
\item We convert gpm to kmpl (kilometers per liter):
\[
\frac{1\text{ gal}}{30\text{ mi}}\times\frac{3.8\text{ L}}{1\text{ gal}}\times\frac{1\text{ mi}}{1.6\text{ km}}=0.079\frac{\text{km}}{\text{L}}
\]
\end{enumerate}
\item If a car can travel 30 miles per gallon of gas, how many gallons of
gas does the car need to travel one mile? Show all work and units,
cancelling units when needed. 
\begin{enumerate}
\item It needs $1/30=0.0\overline{3}$ gallons per mile.
\end{enumerate}
\item If a car can travel 30 miles per gallon of gas, how many liters of
gas does the car need to travel one kilometer?~Show all work and units,
cancelling units when needed. 
\begin{enumerate}
\item We can just take the reciprocal of the fraction from (1):
\[
\frac{1\text{ L}}{0.079\text{ km}}=12.69\frac{\text{km}}{\text{L}}
\]
\end{enumerate}
\item If you were looking to buy a new car, what, if anything, does reading
Section 2.2 change about what you'd consider before buying? Explain.
\end{enumerate}
\end{xca}


\section{Units \& Unit Conversions}

\subsection{Tips \& tricks for unit conversion}
\begin{itemize}
\item Always write out the units along with the numbers, putting ``per''
units in the denominator and primary units in the numerator.
\item It's useful to convert units when existing units make your answer
too small to be properly understood (e.g. converting $.0062$ prisoners
per capita below to $620$ prisoners per $100,000$ people.)
\item Write out units at each step {[}demonstrate with conversion suggestion
from the class{]}
\end{itemize}
\begin{xca}
\textbf{(Worksheet) Measurement \& Unit Conversions}
\begin{enumerate}
\item The reading covered one of the conundrums about the way we report
fuel efficiency. Here we practice these types of calculations to make
sure the ideas are clear. If you assume both vehicles travel the same
distance each year, which option will save the most gas? If you assume
the Honda travels twice as far as the Toyota each year, which option
will save the most gas?
\begin{itemize}
\item Option 1: Replace an old Toyota truck (that gets 17 mpg) with a new
one getting 21 mpg
\begin{itemize}
\item old Toyota: 
\[
10,000\text{ miles}\times\frac{1\text{ mile}}{17\text{ gallons}}=588.24\text{ gallons}
\]
\item new Toyota: 
\[
10,000\times\frac{1}{21}=476.19\text{ gallons}
\]
\item difference: $588.24-476.19=112.05$ gallons
\end{itemize}
\item Option 2: Replace an old Honda Civic (that gets 32 mpg) with a new
one getting 40 mpg.
\begin{itemize}
\item old Civic:
\[
10,000\text{ miles}\times\frac{1\text{ mile}}{32\text{ gallons}}=312.5\text{ gallons}
\]
\item new Civic: 
\[
10,000\times\frac{1}{40}=250\text{ gallons}
\]
\item difference: $312.5-250=62.5$ gallons
\end{itemize}
\item The first option will save more gas if the cars travel the same distance
(assume $10,000$ mi) each year.
\begin{itemize}
\item Note that we're increasing the mpg by $4$ in the first question and
$8$ in the second, yet the first option saves more money.
\end{itemize}
\item If the Honda travels twice as far, say $20,000$ mi, then the old
Civic uses $312.5\times2=625$ gallons and the new Civic uses $250\times2=500$
gallons, a savings of $125$ gallons. So in this case replacing the
Honda makes more sense.
\end{itemize}
\item In the US, roughly $2.2$ million people were in prison (federal,
state or local) in $2018$. In Russia, that number is just under 900,000.
Why would it be misleading to say that Russia imprisons half of the
people the US does? What units would make a comparison more fair?
Look up incarceration rates for various countries. 
\begin{enumerate}
\item We need to look at prisoners \emph{per capita}: Russia has $144.3$
million people so 
\[
\frac{900,000\text{ prisoners}}{144,300,000\text{ people}}=.0062\text{ prisoners per capita}
\]
\item It's more useful to think in bigger numbers: that's $620$ prisoners
per $100,000$ people.
\item In the US, in contrast, there are
\[
\frac{2\text{ million prisoners}}{300\text{ million people}}=.0066\text{ prisoners per capita}.
\]
That's $667$ prisoners per $100,000$ people. So Russia and the US
have similar incarceration rates.
\item \includegraphics{pasted1}
\end{enumerate}
\item Convert the following to units that make more sense to you:
\begin{itemize}
\item Capetown, South Africa target water usage: $87$ liters / day 
\begin{itemize}
\item $23$ gal/day
\end{itemize}
\item German Gas Prices: $1.42$ Euros / L 
\begin{itemize}
\item $\$6.29$ per gallon
\end{itemize}
\item South African VW\textquoteright s efficiency: $5.1\text{L}/100\text{km}$
\begin{itemize}
\item $46.12$ mpg
\end{itemize}
\item Recycled Paper: $1379$ RMB/tonne
\begin{itemize}
\item $\$0.09$/pound
\end{itemize}
\end{itemize}
\item Put the following liquids in order from cheapest to most expensive
(per unit volume) by converting all prices to a single standard. Explain
the widely different prices for these liquids.
\begin{itemize}
\item Five Wives Vodka (\$17.99 / 750 mL)
\begin{itemize}
\item $\$24$/L
\end{itemize}
\item Oil (West Texas Intermediate crude) (\$69.61 / barrel) 
\begin{itemize}
\item $\$69.61/158.987\text{ L}=\$0.43$ per liter
\end{itemize}
\item Milk (\$3.79 / gallon)
\begin{itemize}
\item $\$1.00$/L
\end{itemize}
\item Mount Olympus bottled water (\$4.49 / 12 L) 
\begin{itemize}
\item $\$0.37/L$
\end{itemize}
\item Pure Vanilla Extract (\$6.29 / 59 ml)
\begin{itemize}
\item \$$106.61/L$
\end{itemize}
\item The ordering is
\begin{enumerate}
\item Bottled water
\item Oil
\item Milk
\item Vodka
\item Vanilla extract (kinda surprising to me how oil is relatively cheap,
but this is almost certainly 
\end{enumerate}
\end{itemize}
\item (from mpgillusion.com) Assume that a person drives 10,000 miles per
year and is contemplating changing from a current vehicle to a new
one. First, without doing any calculations (just using intuition and
guesswork), rank the following five pairs of old and new vehicles
in order of their benefit to the environment (i.e., which new car
would reduce gas consumption the most compared to the original car).
Use 1 for the largest savings and 5 for the smallest savings. Then
verify your answers through calculation. 
\begin{enumerate}
\item 18 MPG to 28 MPG
\item 16 MPG to 20 MPG 
\item 22 MPG to 24 MPG 
\item 42 MPG to 48 MPG 
\item 34 MPG to 50 MPG 
\end{enumerate}
\item Convert each answer above from MPG to GPM. Do the GPM numbers reflect
the true benefits in changing vehicles? Explain.
\end{enumerate}
\end{xca}


\end{document}
