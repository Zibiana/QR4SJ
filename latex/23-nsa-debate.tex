%% LyX 2.3.6.2 created this file.  For more info, see http://www.lyx.org/.
%% Do not edit unless you really know what you are doing.
\documentclass[oneside,english]{amsart}
\usepackage[T1]{fontenc}
\usepackage{url}
\usepackage{amsthm}

\makeatletter
%%%%%%%%%%%%%%%%%%%%%%%%%%%%%% Textclass specific LaTeX commands.
\numberwithin{equation}{section}
\numberwithin{figure}{section}
\theoremstyle{plain}
\newtheorem{thm}{\protect\theoremname}
\theoremstyle{definition}
\newtheorem{xca}[thm]{\protect\exercisename}

\makeatother

\usepackage{babel}
\providecommand{\exercisename}{Exercise}
\providecommand{\theoremname}{Theorem}

\begin{document}
\title{WCSAM 206 12.5 - The NSA}
\maketitle

\section{How does the NSA hack our emails?}
\begin{xca}
\textbf{(Reading question; discuss in class; students have watched
video/submitted responses before class) }Please watch the video at
\url{https://www.youtube.com/watch?v=ulg_AHBOIQU} and the supplement
at \url{https://www.youtube.com/watch?v=1O69uBL22nY} and consider
the following questions:
\begin{enumerate}
\item What are elliptic curves?
\begin{enumerate}
\item Equations with two variables like $y^{2}=x^{3}-3x+5$, modulo some
large prime number.
\end{enumerate}
\item Explain how elliptic curves are used for cryptography, assuming you're
talking to someone who doesn't know any cryptography or anything about
elliptic curves.
\begin{enumerate}
\item Using curves like $y^{2}=x^{3}-3x+5\mod p$ for some prime $p$ to
produce points, near the curve, which seem completely random (we extract
pseudorandom bits from these points)
\item Start with some ''seed'' that you choose yourself
\item Need two solutions $P,Q$, one to start the cycle and one to scramble
the output of the cycle. These solutions become the steps in the procedure.
\item These will become the ''pseudorandom bits'' that will be used as the
key(s) for some cipher later on.
\end{enumerate}
\item What is a backdoor, and how did the NSA build a backdoor into their
recommended elliptic curve solutions?
\begin{enumerate}
\item A backdoor is a hidden way to break a code by using additional information
about the cipher that is not commonly known.
\item Knowing the relationship between $P$ and $Q$ is the backdoor that
allows the NSA to determine which ''pseudorandom bits'' will be spit
out of the algorithm. The NSA reverse-engineered the chosen points
$P,Q$ to have a specific relationship known only to the NSA.
\end{enumerate}
\item Do you agree with Frenkel's statement that ``mathematics can be used
for good, but it can also be used for evil''? What does he mean by
this?
\begin{enumerate}
\item Math has the opportunity for misuse, e.g. for the government to read
our private information
\end{enumerate}
\item How did the NSA persuade companies to use their preferred algorithm
for elliptic curve cryptography? 
\begin{enumerate}
\item It published a document encouraging people to use their chosen solutions,
which are hard to find by hand, but which the NSA provided.
\item The NSA may have paid companies to use the 'cracked' algorithm.
\end{enumerate}
\item Do you believe the NSA's reading of private emails was necessary for
national security? How much privacy do you think it's appropriate
to sacrifice for security?
\item Who do you think is most to blame for the NSA's abillity to read private
emails?
\end{enumerate}
\end{xca}


\section{Debate: Is the NSA's Surveillance of Americans Justified?}
\begin{xca}
Watch the video at \url{https://www.youtube.com/watch?v=Rs27d3QZhz8},
(ok well probably the John Oliver at \url{https://www.youtube.com/watch?v=XEVlyP4_11M};
FISA in 2020 at \url{https://www.nytimes.com/2021/04/30/us/politics/national-security-surveillance-pandemic.html})
thinking about the following questions:
\begin{itemize}
\item What is the history of the NSA? How did the US use cryptography and
ciphers historically?
\item The video is clearly outdated; what do you think has changed since
the video was made? Why am I showing you such an outdated video?
\item Do you think the NSA's surveillance of Americans is required to protect
us?
\item Is it legal?
\item What amount of liberty/privacy is it OK to sacrifice for increased
security?
\item Do you trust the NSA with your information?
\item If an administration comes to power which doesn't like people with
a certain religious or political belief, how could the surveillance
already being done be used?
\item Are people like Edward Snowden heroes? Traitors? Do they deserve punishment?
\end{itemize}
\end{xca}

~
\begin{xca}
Form 2 groups, ``yes'' and ``no'', at random. Split each group
in half: half to record what's said, half to debate. Use these rules
for debate:
\begin{itemize}
\item No personal criticism allowed. 
\item The more the better. The more ideas proposed, the more likely something
useful will happen. A premium will be placed on unusual or unique
ideas or suggestions.
\end{itemize}
Instructor roles:
\begin{itemize}
\item Write down interesting points and bring them up again when appropriate
\item Only intervene if things get far off-topic or unproductive. 
\item Tolerate brief periods of silence -- that's when students are thinking!
\item Redirect questions to the class.
\item At the end, ask students to summarize the discussion.
\end{itemize}
\end{xca}


\end{document}
