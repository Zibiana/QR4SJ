%% LyX 2.3.6.2 created this file.  For more info, see http://www.lyx.org/.
%% Do not edit unless you really know what you are doing.
\documentclass[oneside,english]{amsart}
\usepackage[T1]{fontenc}
\usepackage{geometry}
\geometry{verbose,tmargin=1in,bmargin=1in,lmargin=1in,rmargin=1in}
\usepackage{babel}
\usepackage{url}
\usepackage{amstext}
\usepackage{amsthm}
\usepackage{graphicx}
\usepackage[unicode=true,pdfusetitle,
 bookmarks=true,bookmarksnumbered=false,bookmarksopen=false,
 breaklinks=false,pdfborder={0 0 1},backref=false,colorlinks=false]
 {hyperref}

\makeatletter
%%%%%%%%%%%%%%%%%%%%%%%%%%%%%% Textclass specific LaTeX commands.
\numberwithin{equation}{section}
\numberwithin{figure}{section}
\theoremstyle{plain}
\newtheorem{thm}{\protect\theoremname}
\theoremstyle{definition}
\newtheorem{xca}[thm]{\protect\exercisename}

\makeatother

\providecommand{\exercisename}{Exercise}
\providecommand{\theoremname}{Theorem}

\begin{document}
\title{WCSBS 220 8 - Climate Change}
\maketitle

\section{Sea Level Change and Function Composition}
\begin{itemize}
\item Although most carbon dioxide emissions have been produced by wealthier
industrial societies, it is poorer nations which will be impacted
first and worst by any resulting change in climate. This is environmental
injustice on an international scale.
\item One example of this is the dramatic impact sea level rise is expected
to have on subsistence fisher-folk living on low lying islands. In
this module, you'll work with a linear function and a quadratic function
representing the sea level rise in the Republic of Maldives and create
a simplified model of the geography of the Maldives (a triangular
prism) to study the effects of sea-level rise in the region. 
\item After you compute the geographical effects of sea-level rise, you'll
research and write about how this impacts livelihoods and society
in the Maldives and similar island nations.
\item Since the beginning of the industrial age, the concentration of greenhouse
gases in the Earth\textquoteright s atmosphere has been increasing
{[}19{]}. The most common greenhouse gas is carbon dioxide (CO2). 
\item Sea-level rise is already affecting the coastline of the United States,
with significant impacts projected later in the century for major
cities such as New York and Miami. 
\end{itemize}
\begin{xca}
\textbf{(Reading Question) }Please watch the first four minutes of
the United Nations Development Programme video below, on the impacts
of sea level rise on the pacific islands of Kiribati, then answer
the following questions in complete sentences, showing your work.
Bring your work to class on the date below for discussion.
\begin{enumerate}
\item What are your first impressions from the video? 
\item What is causing many areas of Kiribati to become uninhabitable?
\begin{itemize}
\item The reason the sea-level is rising is due to rising global average
temperatures. Scientists now believe that the global average temperature
will increase by about 2.5 degrees Celsius by $2100$ as a result
of increasing concentrations of greenhouse gases such as $CO_{2}$
and methane in the atmosphere.
\begin{itemize}
\item To convert a \textbf{change} in global average temperature to Fahrenheit,
multiply by $2$ (roughly). So this is a $5$ degree Fahrenheit increase,
turning a ``pleasant'' 85-degree spring day to a $90$-degree day
on average.
\end{itemize}
\item This may not sound like much, but current global average temperatures
are only $3$ to $5^{\circ}$ Celsius higher than in the last ice
age, during which sea levels were about $122$ m lower than they are
today. 
\item Greenhouse gases warm the earth by preventing some of the sun\textquoteright s
energy from escaping back to space. 
\item The primary increases in concentrations of greenhouse gases in the
atmosphere are from the burning of fossil fuels, transport, agriculture,
and changes in land usage such as deforestation {[}6, p. 45-56{]}. 
\item As Walker and King note in {[}19, p. 188{]}, \textquotedblleft The
world\textquoteright s major industrialized countries . . . have been
responsible for almost all of the current climate problem; they gained
their wealth and advanced state of development largely by exploiting
cheap fossil fuels at an early stage.\textquotedblright{} 
\item However, it is not these wealthier industrialized countries which
will experience the most severe effects of the climate changes brought
about by the increased concentrations of CO2, but rather poorer nations
such as those which are heavily dependent on agriculture, tourism,
and fishing {[}3, 6, 7, 16{]}. Scientists believe that the global
average temperature will increase by about $2.5^{\circ}$ C by $2100$
as a result of increasing concentrations of greenhouse gases in the
atmosphere.
\item Rising sea-levels will affect many aspects of life for people living
in poverty and in coastal areas. Some of these aspects are: erosion
{[}6{]}, saline intrusion into fresh (drinking or agricultural) water
{[}6{]}, loss of ancestral home lands {[}6{]}, loss of revenue from
tourism and fishing {[}11{]}, loss of revenue and utility of ports
{[}6{]}, increased incidences of mental health issues {[}6{]}, coastal
flooding {[}3, 19{]}, increased disease incidence (especially due
to water-born pathogens with increased flooding) {[}3, 16{]}, social
and political upheaval {[}6{]}, and decreased crop productivity {[}6,
19{]}.
\end{itemize}
\item Name two factors making Kiribati particularly vulnerable to the effects
of climate change. 
\begin{enumerate}
\item Kirabati is no more than four metres high at its highest point (about
$12$-13 feet)
\item 100 percent of the population lives within one kilometre of the coast
(about half a mile), making this nation one of the most vulnerable
to the effects of global warming
\end{enumerate}
\item The Intergovernmental Panel on Climate Change (IPCC) predicts between
0.28 and 0.98 meters of sea level rise between $2009$ and 2100 due
to thermal expansion and the melting of ice sheets in Greenland and
Antarctica. If the sea level rises at the lower level predicted by
the IPCC, how many inches does it rise per year? 
\[
\frac{0.28\text{ meters}}{2100-2009\text{ years}}\times\frac{39.37\text{ in}}{1\text{ meter}}=\frac{0.28(39.37)\text{ meters}}{91\text{ years}}=0.12\text{ in/year}.
\]
\item We'll discuss a case study involving another Pacific island chain,
the Maldives, where more reliable data is available on sea level.
Since the actual shape of the Maldives is complicated, we will use
a simpler shape to compute with and assume this is a good stand-in
for the real islands. We will pretend the nation is one triangular
prism. The width of the base is 1 km and the length of the base is
320 km. The height of the prism is 2.4 m. Convert all units to kilometers
and draw a picture labeled with measurements in kilometers. What is
the map area of the prism in square kilometers? By map area, in this
project, we mean the area of the rectangle you would see by looking
down on the prism from directly above it. 

\includegraphics[scale=0.5]{Maldives_map_area}
\item How close to the true area and coastline measurements is this approximation?
\begin{enumerate}
\item The actual area of the Maldives is $297.8$ square km.
\item The coastline measurements of our prism would be the perimeter of
the base, which is $2(320)+2(1)=642$ km. 
\item The actual coastline of the Maldives is $644$ km long--a very close
approximation!
\item It could have been closer, but for simplicity, the lengths were rounded
to the nearest kilometer. One can show the average height of the prism
is 1.2 m, which is a little lower than the average height of the true
Maldives which is approximately 1.5 m.
\end{enumerate}
\end{enumerate}
\end{xca}

\begin{itemize}
\item In this module, I have chosen to focus on the Maldives because it
was the lowest-lying nation for which sufficient data were available.
\item In 2010, the USA\textquoteright s per capita CO2 emissions were 17.6
metric tons, but the Maldives per capita CO2 emissions were 3.3 metric
tons {[}20{]}. 
\item Unfortunately, despite not being a major contributor to the problem,
the Maldives is an example of a poor low-lying country which will
be hit soon and hard by the effects of global climate change including
sea-level rise. 
\item The Maldives is located south-southwest of India. It consists of approximately
1,190 islands and is the lowest country in the world with a maximum
elevation of 2.4 m. 
\item Before attempting the worksheet, we need two problems' worth of practice
with applied function composition.
\end{itemize}
\begin{xca}
\textbf{(RQ) }Consider the following data and the linear model that
fits it (source: NOAA.gov): CO2vsTemp.png. The equation for the linear
model is $y=0.019x+4.76$. Answer the following questions in complete
sentences with units.
\begin{enumerate}
\item What are the units for the slope? 
\begin{itemize}
\item Since the slope is computed via $m=\frac{\Delta y}{\Delta x}$, the
units of the slope are the units of the $y$ variable divided by the
units of the $x$-variable. 
\item Hence, the units for the slope are $^{\circ}\text{C}/\text{ppm}$.
\end{itemize}
\item What does the slope mean in this particular situation? 
\begin{itemize}
\item For every increase in the CO2 concentration at Mauna Loa by $1$ ppm,
the average continental US temperature will increase by about $0.019^{\circ}$
C.
\end{itemize}
\item What are the units for the $y$-intercept? 
\begin{itemize}
\item The units for the $y$-intercept are the same as the units for the
$y$-variable, since the $y$-intercept is just the $y$-value when
$x=0$.
\item Hence, the units for the $y$-intercept are $^{\circ}C$.
\end{itemize}
\item What does the $y$-intercept mean in this particular situation? 
\begin{itemize}
\item The $y$-intercept would represent the average continental US temperature
if the atmospheric CO2 concentration at Mauna Loa were reduced to
$0$ ppm. (This is impossible; some atmospheric CO2 is inevitable.)
\end{itemize}
\item Use the linear model to predict what the continental US temperature,
in degrees Celsius, would be if the CO2 count at Mauna Loa were 670
ppm. (This is the expected CO2 count in 2100 if the Paris climate
accords of 2015 are followed to the letter.) Do you feel confident
in your prediction? Why or why not?
\begin{itemize}
\item We plug in $x=670$ into the linear model to get theat the predicted
continental average US temperature would be
\[
0.019(670)+4.76=17.49^{\circ}\text{ C}\approx63.48^{\circ}\text{ F}.
\]
\item I do not feel confident in this prediction because extrapolation far
outside of the range of available data using a regression line is
generally unfounded.
\end{itemize}
\end{enumerate}
\end{xca}

~
\begin{xca}
The radius of an oil slick, found at \url{https://goo.gl/9br7Cg},
expands at a rate of $2$ m/min. Follow the steps to answer the following
questions.
\begin{enumerate}
\item Play with the applet as much as needed to get a sense of how the area
is changing as the radius changes. What kind of function does it seem
like is being traced out? Does this match what you know about circles?
\begin{enumerate}
\item A quadratic (parabola) function appears to be traced out. This matches
the fact that the area of a circle follows the equation $A=\pi r^{2}$.
\end{enumerate}
\item Write a formula for the radius of the oil slick after $t$ min, assuming
that the slick started with a radius of $1$ m.
\begin{itemize}
\item The radius of the circle at time $t$ is $r(t)=1+2t$ m.
\item This makes sense because the slope, $2$, represents the rate of change
of this linear function.
\end{itemize}
\item Use function composition to write a formula for the area of the slick
after $t$ min, assuming that the slick started with a radius of $1$
m.
\begin{itemize}
\item The area of the circle at time $t$ satisfies the equation $A(t)=\pi[r(t)]^{2}$.
\item Thus, 
\begin{align*}
A(t) & =\pi(1+2t)^{2}=\pi(1+4t+4t^{2})\\
 & =\pi+4\pi t+4\pi t^{2}\text{ m}^{2}.
\end{align*}
\end{itemize}
\end{enumerate}
\end{xca}

~
\begin{xca}
\textbf{{[}Worksheet: Maldives{]}}
\end{xca}


\end{document}
